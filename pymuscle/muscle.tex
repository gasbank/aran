\documentclass[a4paper,10pt]{article}
%\usepackage[landscape]{geometry}
\usepackage{color}
\usepackage{amsmath}
\usepackage{graphicx}

\definecolor{orange}{rgb}{1,0.5,0}
\definecolor{opaqueblue}{rgb}{0.5,0.5,1.0}
\begin{document}

%%%%%%%%% TITLE
\title{A biomechanical approach on biped dynamics}

\author{Geoyeob Kim\\
Computer Graphics Lab., KAIST
}
\date{2010}
\maketitle



\section{Introduction}
A character animation technique related to bipedal characters is
a very interesting research topic since it is an essential component
of every movie and video game. By designing and controlling the character
With a consistent dynamics model, a realistic and physically correct
motion can be synthesized automatically instead of key-framing the motion
manually. There has been a lot of approaches related to
synthesize the character motion with a physically-simulated manner.
Most of these work are based on a model with articulated figures and
actuated joints. In this model the character's body segments are considered
as a set of rigid bodies. Joints are connecting these rigid bodies.
This kind of model is
vastly used since it is suitable to formulate the Lagrangian equations of motion
using the generalized coordinates which helps to reduce the number of degree-of-freedom (DoF).
The joints used in the model are considered as ideal rigid joints, which means
each of them connects the two rigid bodies with a certain DoF. For example,
a hinge joint which has 1-DoF is used at knees or elbows and a ball joint
which has 3-DoF is used at hips. Actuation on the joints are occurred
in their respective DoF. So it is natural to call these kind of joints
\emph{actuated motor joints}. Various animations can be created by controlling these motor joints properly. In robotics this kind of approach is much more
common. The main interest of creating motions with dynamics model is focused on how to design the motor joint controllers.


The motor joints are inherently nonsense in the perspective of
physical realism. No human being has that kind of motors in their joints.
Instead, they control their legs or arms by pulling bones using attached
muscles. Unfortunately, the whole mechanism is quite complicated to understand completely.
However the structure of human muscle fibers is well-known and there exist the biomechanical experiments concerning the most basic human actions such as walking and jumping.[REF NEEDED]

This thesis introduces a biomechanical approach on biped dynamics. Basically,
we introduces muscle fibers as the only actuating elements
of the biped and get rid of all rigid motor joints existed in the previous biped
model. Therefore joints are not rigid anymore. Instead, in the place of joints,
we use ligaments to hold the bodies firmly but not rigidly. This substitution
simulates human cartilages. The cartilages provide a shock absorption
system to sustain stability when external or internal disturbances exist.
We control the rigid bodies with muscle fibers which are modeled as
spring-damper system. Every muscle fiber has a linear actuator and
motions are created by controlling the actuator. Instead of controlling
the bodies with motor joints, it is highly effective in energy concern
since torques are created more easily by pulling the bones with the muscle fiber on a distant place from an axis of rotation such as hips or ankles.

Human gait analysis is also very extensively studied area.[REF] It deals
with the mechanisms of human locomotion by throughly experimenting and measuring
the subjects. They classified human walking cycle in which a division
has its own objectives and features.
In computer animation, the most of work so far did not
exploit this kind of research results. They rely on optimization-based
techniques such as a motion tracker with a linear quadratic regulator (LQR)
or a quadratic programming optimizer attained by heuristic observations
such as 'minimize the actuation energy all the time' or 'follow the
reference trajectory with all your best'. They blindly seek the
natural human behavior of locomotion with these macroscopic objectives.
Although it seemed to have succeeded for generating nice-looking motion,
it can be enhanced by introducing biomechanical knowledge into the both of
a dynamics model and an optimization scheme.

The optimization-based approach is also used in this thesis. However,
we take gait phases analyzed in [Perry] into account. Given motion
capture data containing locomotion of human, we classify the motion
sequence automatically into respective phases. For each phase,
an optimization formulation is built to calculate muscle actuation forces
of our biped model. The passive and active muscles are explicitly defined
with respect to a corresponding phase.
This helps our approach speed-up the simulation
and also gives more energy-efficient and biomechanically-correct result.
[CLEAR MENTION OF CONTRIBUTION IS NEEDED]



\section{Related work}

\section{Problem definition}

\section{Biped dynamics}

Rigid body dynamics are commonly used in physical character model.
Our character model consists of 7 rigid bodies. One of these is
head-trunk-arms (HAT) body. By using HAT instead of every upper body
part, we can concentrate our concern to legs which are essential to
walking. The forward dynamics algorithm gives the following equation:


\begin{equation}
\ddot\chi =\mathbf{M}^{-1}(\chi) ( \tau - C(\chi,\dot\chi) )
\end{equation}
where $\mathbf{M}$ is a mass matrix and $\chi$ is a linear and angular position
vector and $\dot\chi$, $\ddot\chi$ are its first and second time derivatives,
respectively. $C$ represents all the other forces except the actuated and
external forces represented as $\tau$. For the case of freely moving single
rigid body with 6-DoF, $\mathbf{M}$ will be $6\times 6$ matrix and $\chi$,
$\tau$ and $C$ will be a 6-dimensional vector.

We define a discrete-time state vector $Y$. For brevity, we use $Y^{(l)}$
to indicate a discrete-time value of $Y$ at $l$-th time step $Y(t_l)$.
We use an explicit Euler integration
scheme using both $Y$ and the equations of motion to yield a complete
forward dynamics formulation.


\begin{equation}
Y^{(l)} =
\left[ {\begin{array}{cc}
 \chi^{(l)}   \\
 \dot\chi^{(l)}   \\
 \end{array} } \right]
\end{equation}


\begin{equation}
f(Y^{(l)})=\dot{Y}^{(l)}
=
\left[ {\begin{array}{cc}
 \dot\chi^{(l)}   \\
 \ddot\chi^{(l)}   \\
 \end{array} } \right]
 =
\left[ {\begin{array}{cc}
 \dot\chi^{(l)}   \\
 \mathbf{M}^{-1}(\chi^{(l)}) ( \tau^{(l)} - C(\chi^{(l)},\dot\chi^{(l)}) )   \\
 \end{array} } \right]
\end{equation}

\begin{equation}
Y^{(l+1)}=Y^{(l)}+hf(Y^{(l)})
\end{equation}

\subsection{Rigid body dynamics}

\subsection{Muscle fiber model}

There are many variants of muscle model used in biomechanical area range
from a simple spring model to a time-varying nonlinear model. Although
a more sophisticated model gives us a more realistic behavior, unnecessarily
complicated models will make the biped hard to simulate. For a compromise,
we use one of the simplest time-invariant spring-damper model described in [Shadmehr].
Basically, the model consists of two springs and a viscous damper.
The damper and the spring are connected in parallel and one spring is
connected in serial way. The active component in which our controller
can actuate resides in parallel side. A bundle of muscle fibers can be
categorized according to their structure and we treat this bundle as
a single fiber to simplify the model. For each muscle fiber, we need three
parameters: a serial spring constant $k_{se}$, a parallel spring constant $k_{pe}$,
a viscosity $b$ and a rest length $x_{r}$. The model can be formulated
as a first-order ordinary differential equation of a tension $T$ exerted
at the end of the fiber:

\begin{equation}\label{muscle-ode}
\dot{T} = \frac{k_{se}}{b} \left( k_{pe}(x-x_{r})+b\dot{x}-\left(1+\frac{k_{pe}}{k_{se}}\right)T+A   \right)
\end{equation}
where $x$ is the length of the fiber and $A$ is applied force in the active
component. Note that every quantities shown in the equation are scalar values.
Since a muscle fiber always connects two bones (rigid bodies) and the
tension applied to the both of them, we need a clear definition of signed-ness
of the tension $T$. Researchers in biology defined the term $origin$ and $insertion$.
Origin and insertion indicate where a muscle fiber originated from and inserted
to, respectively. If we call the origin and insertion position of the fiber
as $p_{org}$ and $p_{ins}$, the normalized direction of the fiber can be
defined as $\hat{d}=(p_{ins}-p_{org})/||p_{ins}-p_{org}||$. With these quantities, we can think that
the tension $T\hat{d}$ and $-T\hat{d}$ are applied to the origin and insertion body
of the fiber, respectively.

THE INTERPRETATION OF THE EQUATION GOES HERE

\subsection{Whole biped model}

By modeling the bones as rigid bodies and the bunch of muscles as muscle
fibers described so far, we can build the whole biped model. Assume that
we have $n$ rigid bodies and $m$ muscle fibers on the biped. The state vector
$Y$ can be defined:

\begin{equation}
Y = [ p_1, q_1, \dot{p}_1, \dot{q}_1, \cdots ,
      p_n, q_n, \dot{p}_n, \dot{q}_n,
      T_1, \cdots, T_m]
\end{equation}
where $p_i$ and $q_i$ represent linear and angular position of body $i$.
Note that we omitted the transpose operator on vector quantities like $p$,
$q$ and $Y$ for brevity. Time derivate of the state vector can be easily
formulated.

\begin{equation}
\dot{Y} = [ \dot{p}_1, \dot{q}_1, \ddot{p}_1, \ddot{q}_1, \cdots ,
            \dot{p}_n, \dot{q}_n, \ddot{p}_n, \ddot{q}_n,
            \dot{T}_1, \cdots, \dot{T}_m]
\end{equation}
Our concern is mostly concentrated to the quantities like $\ddot{p}$,
$\ddot{q}$ and $\dot{T}$ since the differential equations are formulated
in these variables. The vector $\dot{Y}$ is mixture
of the dynamic formula for rigid bodies and muscle fibers. We can separate
the vector into two parts to help the further analysis and control like the
following:

\begin{eqnarray*}\label{dotY1}
\dot{Y} & = & [ \dot{p}_1, \dot{q}_1, \ddot{\tilde{p}}_1, \ddot{\tilde{q}}_1, \cdots ,
                \dot{p}_n, \dot{q}_n, \ddot{\tilde{p}}_n, \ddot{\tilde{q}}_n,
                0, \cdots, 0] \\
        &   & + [ 0, 0, \ddot{\bar{p}}_1, \ddot{\bar{q}}_1, \cdots ,
                0, 0, \ddot{\bar{p}}_n, \ddot{\bar{q}}_n,
                \dot{T}_1, \cdots, \dot{T}_m]\\
        & = & \dot{Y}_R + \dot{Y}_Q
\end{eqnarray*}
Since we separate the rigid body parts and muscle fiber parts, if we
assume $\dot{Y}_Q$ equals to 0 we simply simulate the $n$ individual
rigid bodies freely move around the space with rotations. If we develop
this kind of separation process further, we get the equations:

\begin{eqnarray*}\label{dotY2}
\dot{Y}_R & = & [ (\dot{p}_1, \dot{q}_1, \ddot{\tilde{p}}_1, \ddot{\tilde{q}}_1),
                  0_{4x}, (0, \cdots , 0)]\\
          &   & + [ 0_{4x}, (\dot{p}_2, \dot{q}_2, \ddot{\tilde{p}}_2, \ddot{\tilde{q}}_2),
                  0_{4x}, (0, \cdots , 0)]\\
          &   & + \cdots\\
          &   & + [ 0_{4x}, (\dot{p}_n, \dot{q}_n, \ddot{\tilde{p}}_n, \ddot{\tilde{q}}_n),
                  (0, \cdots , 0)]\\
          & = & \sum_{i=1}^{n}{\dot{Y}_{R,i}}\\
\dot{Y}_Q & = & [ 0_{4x}, (0, 0, \ddot{\bar{p}}_{T,1}^{org}, \ddot{\bar{q}}_{T,1}^{org}),
                  0_{4x}, (0, 0, \ddot{\bar{p}}_{T,1}^{ins}, \ddot{\bar{q}}_{T,1}^{ins}),
                  0_{4x}, (\dot{T}_1, 0, \cdots, 0)]\\
          &   & + [ 0_{4x}, (0, 0, \ddot{\bar{p}}_{T,2}^{ins}, \ddot{\bar{q}}_{T,2}^{ins}),
                    0_{4x}, (0, 0, \ddot{\bar{p}}_{T,2}^{org}, \ddot{\bar{q}}_{T,2}^{org}),
                    0_{4x}, (0, \dot{T}_2, 0, \cdots, 0)]\\
          &   & + \cdots\\
          &   & + [ 0_{4x}, (0, 0, \ddot{\bar{p}}_{T,m}^{org}, \ddot{\bar{q}}_{T,m}^{org}),
                    0_{4x}, (0, 0, \ddot{\bar{p}}_{T,m}^{ins}, \ddot{\bar{q}}_{T,m}^{ins}),
                    0_{4x}, (0, \cdots, 0, \dot{T}_m)]\\
          & = & \sum_{i=1}^{m}{\dot{Y}_{Q,i}}\\
\end{eqnarray*}
Here $0_{4x}$ denotes an arbitrary zero vector whose dimension is multiple of four.
Parentheses are added just for clarify the structure.
$\dot{Y}_{R,i}$ represents the first derivative of the state vector of the body $i$
when there is no muscle fiber attached to it. $\dot{Y}_{Q,i}$ denotes the effects of
the muscle fiber $i$ affecting the origin and insertion body. Specifically,
$ \ddot{\bar{p}}_{T,i}^{org} $ and $ \ddot{\bar{q}}_{T,i}^{org} $ are the force and
torque affecting the origin body. This kind of separation gives us a simplified
way to analyze the whole system. For instance, if we set $\dot{Y}_{Q_i}$ to 0 then
the effect of muscle fiber $i$ is entirely removed from the simulation.

\section{Contact model}
There are various kind of techniques to deal with the calculation of
contact forces between rigid bodies. The existing algorithms can be
divided into three categories: constraint-based, penalty-based and
impulse-based approaches. In constraint-based methods, the contact
forces are computed by solving an optimization problem such as
linear complementarity problems (LCP). The contact forces are calculated
precisely in this case. It is not well suited in the environments
which contain many contact points. Penalty-based methods compute
the contact forces by considering a contact as a spring-damper.
It is scalable and easy to implement. However it requires very short
integration step especially at collisions and in most cases nonpenetration
conditions are violated. Impulse-based methods model contacts
as successive collisions instead of persisting contacts.

Since our simulation environment contains a small number of contact points
and the precise contact forces helps the physical realism in animation,
we choose a time-stepping method introduced in [STEWART, ANITESCU].

Contact forces are confined in the Coulomb friction cone


\section{Implicit integration}

Most of rigid body systems can be simulated using an explicit Euler method
as far as there is no stiff component exists. The explicit integration
allows us a simple implementation and fast computation of a next state,
however, it has the potential of instability when the time step is larger
then a certain threshold. The threshold is mainly determined by a time
constant of a given differential equation we need to integrate. In our case, the time
constant is dominated by muscle fibers because they consists
of highly stiff springs (\ref{muscle-ode}). Even if we use very small time step
such as $h=10^{-6}$, our system diverges very quickly with the explicit
integration. To make the simulation stable, we used an implicit integration
method.

\begin{equation}
Y^{(l+1)}=Y^{(l)}+hf(Y^{(l+1)})
\end{equation}
That is, we want to find the next state $Y^{(l+1)}$ where we
reverse the time step by $-h$ from $Y^{(l+1)}$, we get to the current state
$Y^{(l)}$. In most cases, $f$ is a nonlinear function of $Y$, so we need to
linearize it to find closed form solution of $Y^{(l+1)}$. If we define a
variable $\Delta Y = Y^{(l+1)}-Y^{(l)}$, it can be calculated using the
following equation:

\begin{equation}
\Delta Y = \left(  \frac{1}{h}\mathbf{1} - {\frac{\partial f}{\partial Y} \bigg|_{Y=Y^{(l)}}}\right)^{-1} f(Y^{(l)})
\end{equation}
where $\mathbf{1}$ is an identity matrix. We need an analytic form of
Jacobian $\partial f / \partial Y$ to calculate the next state in the
implicit integration. From (\ref{dotY1}) and (\ref{dotY2}), we can see that
Jacobian also can be calculated in separated manner.


\begin{align*}
\frac{\partial f}{\partial Y}
        & = \frac{\partial\dot{Y}}{\partial Y}\\
        & = \frac{\partial\dot{Y}_R}{\partial Y} + \frac{\partial\dot{Y}_Q}{\partial Y}\\
        & = \sum_{i=1}^{n}\frac{\partial\dot{Y}_{R,i}}{\partial Y} + \sum_{i=1}^{m}\frac{\partial\dot{Y}_{Q,i}}{\partial Y}
\end{align*}

\includegraphics[width=120mm]{testplot}

\section{A concrete example}

To explain the dynamics model in more specific manner, let us consider
a concrete example consists of three rigid bodies $R_1$, $R_2$ and $R_3$
and three muscle fibers $F_1$, $F_2$ and $F_3$. $R_1$ and $R_2$ are connected
by both of $F_1$ and $F_2$ whereas $F_3$ connects $R_2$ and $R_3$. The
origin and insertion of muscle fibers are described in Figure xx.

A 45-dimension state vector $Y$ is

\begin{align*}
Y = {} & [p_1^x, p_1^y, p_1^z, q_1^w, q_1^x, q_1^y, q_1^z, \dot{p}_1^x, \dot{p}_1^y, \dot{p}_1^z, \dot{q}_1^w, \dot{q}_1^x, \dot{q}_1^y, \dot{q}_1^z,\\
       &  p_2^x, p_2^y, p_2^z, q_2^w, q_2^x, q_2^y, q_2^z, \dot{p}_2^x, \dot{p}_2^y, \dot{p}_2^z, \dot{q}_2^w, \dot{q}_2^x, \dot{q}_2^y, \dot{q}_2^z,\\
       &  p_3^x, p_3^y, p_3^z, q_3^w, q_3^x, q_3^y, q_3^z, \dot{p}_3^x, \dot{p}_3^y, \dot{p}_3^z, \dot{q}_3^w, \dot{q}_3^x, \dot{q}_3^y, \dot{q}_3^z,\\
       & T_1, T_2, T_3]^T
\end{align*}
where all variables are scalers. You may notice that we used quaternions
to parameterize the rotation of rigid bodies. This allows us a freedom
from the gimbal lock problem by sacrificing dimensional complexity. From
now on, we will denote the three or four dimensional vectors is its shorthand
form, i.e., $p_i=[p_i^x, p_i^y, p_i^z]^T$ and $q_i=[q_i^w, q_i^x, q_i^y, q_i^z]^T$.

\begin{equation}
\dot{Y}  =  [\dot{p}_1, \dot{q}_1, \ddot{p}_1, \ddot{q}_1,
               \dot{p}_2, \dot{q}_2, \ddot{p}_2, \ddot{q}_2,
               \dot{p}_3, \dot{q}_3, \ddot{p}_3, \ddot{q}_3,
               \dot{T}_1, \dot{T}_2, \dot{T}_3]
\end{equation}
Again we omit the transpose operators at the places where they needed
unless the omission misleads our exposition.


\begin{align*}
\dot{Y}_R = {} & \dot{Y}_{R,1} + \dot{Y}_{R,2} + \dot{Y}_{R,3}\\
         =  {} &  [   \dot{p}_1, \dot{q}_1, \ddot{\tilde{p}}_1, \ddot{\tilde{q}}_1, 0, 0, 0, 0, 0, 0, 0, 0, 0, 0, 0 ]\\
               & + [ 0, 0, 0, 0, \dot{p}_2, \dot{q}_2, \ddot{\tilde{p}}_2, \ddot{\tilde{q}}_2, 0, 0, 0, 0, 0, 0, 0 ]\\
               & + [ 0, 0, 0, 0, 0, 0, 0, 0, \dot{p}_3, \dot{q}_3, \ddot{\tilde{p}}_3, \ddot{\tilde{q}}_3, 0, 0, 0 ]
\end{align*}
\begin{eqnarray*}
\dot{Y}_Q & = & \dot{Y}_{Q,1} + \dot{Y}_{Q,2} + \dot{Y}_{Q,3}\\
          & = &   [ 0, 0, \ddot{\bar{p}}_{T,1}^{org}, \ddot{\bar{q}}_{T,1}^{org}, 0, 0, \ddot{\bar{p}}_{T,1}^{ins}, \ddot{\bar{q}}_{T,1}^{ins}, 0, 0, 0, 0, \dot{T}_1, 0, 0 ]\\
          &   & + [ 0, 0, \ddot{\bar{p}}_{T,2}^{ins}, \ddot{\bar{q}}_{T,2}^{ins}, 0, 0, \ddot{\bar{p}}_{T,2}^{org}, \ddot{\bar{q}}_{T,2}^{org}, 0, 0, 0, 0, 0, \dot{T}_2, 0 ]\\
          &   & + [ 0, 0, \ddot{\bar{p}}_{T,3}^{ins}, \ddot{\bar{q}}_{T,3}^{ins}, 0, 0, 0, 0, 0, 0, \ddot{\bar{p}}_{T,3}^{org}, \ddot{\bar{q}}_{T,3}^{org}, 0, 0, \dot{T}_3 ]
\end{eqnarray*}
\begin{eqnarray*}
\frac{\partial f}{\partial Y}
        & = & \frac{\partial\dot{Y}_{R,1}}{\partial Y} + \frac{\partial\dot{Y}_{R,2}}{\partial Y} + \frac{\partial\dot{Y}_{R,3}}{\partial Y}
              + \frac{\partial\dot{Y}_{Q,1}}{\partial Y} + \frac{\partial\dot{Y}_{Q,2}}{\partial Y} + \frac{\partial\dot{Y}_{Q,3}}{\partial Y}
\end{eqnarray*}


\begin{equation}
y_{i} = [p_i, q_i, \dot{p}_i, \dot{q}_i]
\end{equation}

\begin{equation}
\dot{y}_{R,i} = [\dot{p}_i, \dot{q}_i, \ddot{\tilde{p}}_i, \ddot{\tilde{q}}_i]
\end{equation}

% User-defined macro for printing zero matrix in opaque color.
% Should be used in math blocks.
% Usage) \zm{2,3}
%        \zm{}
\newcommand{\zm}[1]{\ensuremath{ {\color{opaqueblue} 0_{#1} } }}

\begin{equation}
\frac{\partial\dot{Y}_{R,1}}{\partial Y}=
\left[ \begin{array}{cccc}
\partial \dot{y}_{R,1} / \partial y_1 & \zm{14,14} & \zm{14,14} & \zm{14,3}\\
\zm{14,14}                            & \zm{14,14} & \zm{14,14} & \zm{14,3}\\
\zm{14,14}                            & \zm{14,14} & \zm{14,14} & \zm{14,3}\\
\zm{3,14}                             & \zm{3,14}  & \zm{3,14}  & \zm{3,3}\\
\end{array} \right]
\end{equation}

\begin{equation}
\frac{\partial\dot{Y}_{R,2}}{\partial Y}=
\left[ \begin{array}{cccc}
\zm{14,14} & \zm{14,14}                            & \zm{14,14} & \zm{14,3}\\
\zm{14,14} & \partial \dot{y}_{R,2} / \partial y_2 & \zm{14,14} & \zm{14,3}\\
\zm{14,14} & \zm{14,14}                            & \zm{14,14} & \zm{14,3}\\
\zm{3,14}  & \zm{3,14}                             & \zm{3,14}  & \zm{3,3}\\
\end{array}  \right]
\end{equation}

\begin{equation}
\frac{\partial\dot{Y}_{R,3}}{\partial Y}=
\left[ \begin{array}{cccc}
\zm{14,14} & \zm{14,14} & \zm{14,14}                            & \zm{14,3}\\
\zm{14,14} & \zm{14,14} & \zm{14,14}                            & \zm{14,3}\\
\zm{14,14} & \zm{14,14} & \partial \dot{y}_{R,3} / \partial y_3 & \zm{14,3}\\
\zm{3,14}  & \zm{3,14}  & \zm{3,14}                             & \zm{3,3}\\
\end{array}  \right]
\end{equation}


\begin{equation}
\dot{y}_{Q,i}^{org} = [0, 0, \ddot{\bar{p}}_{T,i}^{org}, \ddot{\bar{q}}_{T,i}^{org}]
\end{equation}

\begin{equation}
\dot{y}_{Q,i}^{ins} = [0, 0, \ddot{\bar{p}}_{T,i}^{ins}, \ddot{\bar{q}}_{T,i}^{ins}]
\end{equation}

\begin{equation}
\frac{\partial\dot{Y}_{Q,1}}{\partial Y}=
\left[ \begin{array}{cccccc}
\partial\dot{y}_{Q,1}^{org} / \partial y_1 & \partial\dot{y}_{Q,1}^{org} / \partial y_2 & \zm{14,14} & \partial\dot{y}_{Q,1}^{org} / \partial T_1 & \zm{14,1} & \zm{14,1} \\
\partial\dot{y}_{Q,1}^{ins} / \partial y_1 & \partial\dot{y}_{Q,1}^{ins} / \partial y_2 & \zm{14,14} & \partial\dot{y}_{Q,1}^{ins} / \partial T_1 & \zm{14,1} & \zm{14,1} \\
\zm{14,14}                                 & \zm{14,14}                                 & \zm{14,14} & \zm{14,1}                                  & \zm{14,1} & \zm{14,1} \\
\partial\dot{T}_1 / \partial y_1           & \partial\dot{T}_1 / \partial y_2           & \zm{1,14}  & \partial\dot{T}_1 / \partial T_1           & \zm{}     & \zm{} \\
\zm{1,14}                                  & \zm{1,14}                                  & \zm{1,14}  & \zm{}                                      & \zm{}     & \zm{} \\
\zm{1,14}                                  & \zm{1,14}                                  & \zm{1,14}  & \zm{}                                      & \zm{}     & \zm{} \\
\end{array}  \right]
\end{equation}

\begin{equation}
\frac{\partial\dot{Y}_{Q,2}}{\partial Y}=
\left[ \begin{array}{cccccc}
\partial\dot{y}_{Q,2}^{ins} / \partial y_1 & \partial\dot{y}_{Q,2}^{ins} / \partial y_2 & \zm{14,14} & \zm{14,1} & \partial\dot{y}_{Q,2}^{ins} / \partial T_2 & \zm{14,1} \\
\partial\dot{y}_{Q,2}^{org} / \partial y_1 & \partial\dot{y}_{Q,2}^{org} / \partial y_2 & \zm{14,14} & \zm{14,1} & \partial\dot{y}_{Q,2}^{org} / \partial T_2 & \zm{14,1} \\
\zm{14,14}                                 & \zm{14,14}                                 & \zm{14,14} & \zm{14,1} & \zm{14,1}                                  & \zm{14,1} \\
\zm{1,14}                                  & \zm{1,14}                                  & \zm{1,14}  & \zm{}     & \zm{}                                      & \zm{} \\
\partial\dot{T}_2 / \partial y_1           & \partial\dot{T}_2 / \partial y_2           & \zm{1,14}  & \zm{}     & \partial\dot{T}_2 / \partial T_2           & \zm{} \\
\zm{1,14}                                  & \zm{1,14}                                  & \zm{1,14}  & \zm{}     & \zm{}                                      & \zm{} \\
\end{array}  \right]
\end{equation}

\begin{equation}
\frac{\partial\dot{Y}_{Q,3}}{\partial Y}=
\left[ \begin{array}{cccccc}
\partial\dot{y}_{Q,3}^{ins} / \partial y_1 & \zm{14,14} & \partial\dot{y}_{Q,3}^{ins} / \partial y_3 & \zm{14,1} & \zm{14,1} & \partial\dot{y}_{Q,3}^{ins} / \partial T_3  \\
\zm{14,14}                                 & \zm{14,14} & \zm{14,14}                                 & \zm{14,1} & \zm{14,1} & \zm{14,1} \\
\partial\dot{y}_{Q,3}^{org} / \partial y_1 & \zm{14,14} & \partial\dot{y}_{Q,3}^{org} / \partial y_3 & \zm{14,1} & \zm{14,1} & \partial\dot{y}_{Q,3}^{org} / \partial T_3  \\
\zm{1,14}                                  & \zm{1,14}  & \zm{1,14}                                  & \zm{}     & \zm{}     & \zm{} \\
\zm{1,14}                                  & \zm{1,14}  & \zm{1,14}                                  & \zm{}     & \zm{}     & \zm{} \\
\partial\dot{T}_3 / \partial y_1           & \zm{1,14}  & \partial\dot{T}_3 / \partial y_3           & \zm{}     & \zm{}     & \partial\dot{T}_3 / \partial T_3 \\
\end{array}  \right]
\end{equation}


\section{Gait phases}

Locomotion such as walking or running is performed by a periodic way.
A gait cycle or stride contains eight functional patterns(FIGURE xx).
[Perry] These patterns are called \emph{gait phases}. Each phase has its
own features and objectives. For example, during mid-swing and terminal swing
phase, one of primary muscle called 'biceps femoris' and 'semimembranosis' is
activated to extend our swing leg forward.

terminal stance phase one
of the muscle  we will use this fact
to our locomotion controller.

Aha!
Change this...
fdsfsd
Whoooooooooooa~
\section{Controller design}


\section{Result}

\section{Conclusion}


\begin{equation}
\mathbf{A}\vec{\ddot{f}}_b(x) + g(x) = 4x + 11dd
\end{equation}

\begin{eqnarray*}
m\vec{a} & = & \vec{f} \\
         & = & \vec{f}_c+\vec{f}_m+\vec{f}_e \\
         & = & \sum_{i=1}^{n_c}{\vec{f}_{c,i}} + \sum_{i=1}^{n_m}{\vec{f}_{m,i}} + \sum_{i=1}^{n_e}{\vec{f}_{e,i}}. \\
\\
\mathbf{H}\vec\alpha + \vec\omega\times\mathbf{H}\vec\omega
         & = & \vec\tau \\
         & = & \sum_{i=1}^{n_c}{\vec{r}_{c,i}\times\vec{f}_{c,i}} + \sum_{i=1}^{n_m}{\vec{r}_{m,i}\times\vec{f}_{m,i}} + \sum_{i=1}^{n_e}{\vec{r}_{e,i}\times\vec{f}_{e,i}}. \\
\end{eqnarray*}

\begin{equation}
\frac{\mathrm{v1}\, \cos\!\left(\frac{\sqrt{\mathrm{thetasq}}}{2}\right)}{2\, \mathrm{thetasq}} - \frac{\mathrm{v1}\, \sin\!\left(\frac{\sqrt{\mathrm{thetasq}}}{2}\right)}{{\mathrm{thetasq}}^{\frac{3}{2}}} - \frac{3\, \mathrm{v1}\, {\mathrm{v2}}^2\, \cos\!\left(\frac{\sqrt{\mathrm{thetasq}}}{2}\right)}{2\, {\mathrm{thetasq}}^2} - \frac{\mathrm{v1}\, {\mathrm{v2}}^2\, \sin\!\left(\frac{\sqrt{\mathrm{thetasq}}}{2}\right)}{4\, {\mathrm{thetasq}}^{\frac{3}{2}}} + \frac{3\, \mathrm{v1}\, {\mathrm{v2}}^2\, \sin\!\left(\frac{\sqrt{\mathrm{thetasq}}}{2}\right)}{{\mathrm{thetasq}}^{\frac{5}{2}}}
\end{equation}

\begin{equation}
\left(\begin{array}{c} \frac{\mathrm{dv1}\, \sin\!\left(\sqrt{{\mathrm{v1}}^2 + {\mathrm{v2}}^2 + {\mathrm{v3}}^2}\right)}{\sqrt{{\mathrm{v1}}^2 + {\mathrm{v2}}^2 + {\mathrm{v3}}^2}} + \frac{\mathrm{dv1}\, {\mathrm{v1}}^2}{{\left({\mathrm{v1}}^2 + {\mathrm{v2}}^2 + {\mathrm{v3}}^2\right)}^{\frac{3}{4}}} - \frac{\mathrm{dv1}\, {\mathrm{v1}}^2\, \sin\!\left(\sqrt{{\mathrm{v1}}^2 + {\mathrm{v2}}^2 + {\mathrm{v3}}^2}\right)}{{\left({\mathrm{v1}}^2 + {\mathrm{v2}}^2 + {\mathrm{v3}}^2\right)}^{\frac{5}{4}}} + \frac{2\, \mathrm{dv3}\, \mathrm{v2}\, {\sin\!\left(\frac{\sqrt{{\mathrm{v1}}^2 + {\mathrm{v2}}^2 + {\mathrm{v3}}^2}}{2}\right)}^2}{{\mathrm{v1}}^2 + {\mathrm{v2}}^2 + {\mathrm{v3}}^2} - \frac{2\, \mathrm{dv2}\, \mathrm{v3}\, {\sin\!\left(\frac{\sqrt{{\mathrm{v1}}^2 + {\mathrm{v2}}^2 + {\mathrm{v3}}^2}}{2}\right)}^2}{{\mathrm{v1}}^2 + {\mathrm{v2}}^2 + {\mathrm{v3}}^2} + \frac{\mathrm{dv2}\, \mathrm{v1}\, \mathrm{v2}}{{\left({\mathrm{v1}}^2 + {\mathrm{v2}}^2 + {\mathrm{v3}}^2\right)}^{\frac{3}{4}}} + \frac{\mathrm{dv3}\, \mathrm{v1}\, \mathrm{v3}}{{\left({\mathrm{v1}}^2 + {\mathrm{v2}}^2 + {\mathrm{v3}}^2\right)}^{\frac{3}{4}}} - \frac{\mathrm{dv2}\, \mathrm{v1}\, \mathrm{v2}\, \sin\!\left(\sqrt{{\mathrm{v1}}^2 + {\mathrm{v2}}^2 + {\mathrm{v3}}^2}\right)}{{\left({\mathrm{v1}}^2 + {\mathrm{v2}}^2 + {\mathrm{v3}}^2\right)}^{\frac{5}{4}}} - \frac{\mathrm{dv3}\, \mathrm{v1}\, \mathrm{v3}\, \sin\!\left(\sqrt{{\mathrm{v1}}^2 + {\mathrm{v2}}^2 + {\mathrm{v3}}^2}\right)}{{\left({\mathrm{v1}}^2 + {\mathrm{v2}}^2 + {\mathrm{v3}}^2\right)}^{\frac{5}{4}}}\\ \frac{\mathrm{dv2}\, \sin\!\left(\sqrt{{\mathrm{v1}}^2 + {\mathrm{v2}}^2 + {\mathrm{v3}}^2}\right)}{\sqrt{{\mathrm{v1}}^2 + {\mathrm{v2}}^2 + {\mathrm{v3}}^2}} + \frac{\mathrm{dv2}\, {\mathrm{v2}}^2}{{\left({\mathrm{v1}}^2 + {\mathrm{v2}}^2 + {\mathrm{v3}}^2\right)}^{\frac{3}{4}}} - \frac{\mathrm{dv2}\, {\mathrm{v2}}^2\, \sin\!\left(\sqrt{{\mathrm{v1}}^2 + {\mathrm{v2}}^2 + {\mathrm{v3}}^2}\right)}{{\left({\mathrm{v1}}^2 + {\mathrm{v2}}^2 + {\mathrm{v3}}^2\right)}^{\frac{5}{4}}} - \frac{2\, \mathrm{dv3}\, \mathrm{v1}\, {\sin\!\left(\frac{\sqrt{{\mathrm{v1}}^2 + {\mathrm{v2}}^2 + {\mathrm{v3}}^2}}{2}\right)}^2}{{\mathrm{v1}}^2 + {\mathrm{v2}}^2 + {\mathrm{v3}}^2} + \frac{2\, \mathrm{dv1}\, \mathrm{v3}\, {\sin\!\left(\frac{\sqrt{{\mathrm{v1}}^2 + {\mathrm{v2}}^2 + {\mathrm{v3}}^2}}{2}\right)}^2}{{\mathrm{v1}}^2 + {\mathrm{v2}}^2 + {\mathrm{v3}}^2} + \frac{\mathrm{dv1}\, \mathrm{v1}\, \mathrm{v2}}{{\left({\mathrm{v1}}^2 + {\mathrm{v2}}^2 + {\mathrm{v3}}^2\right)}^{\frac{3}{4}}} + \frac{\mathrm{dv3}\, \mathrm{v2}\, \mathrm{v3}}{{\left({\mathrm{v1}}^2 + {\mathrm{v2}}^2 + {\mathrm{v3}}^2\right)}^{\frac{3}{4}}} - \frac{\mathrm{dv1}\, \mathrm{v1}\, \mathrm{v2}\, \sin\!\left(\sqrt{{\mathrm{v1}}^2 + {\mathrm{v2}}^2 + {\mathrm{v3}}^2}\right)}{{\left({\mathrm{v1}}^2 + {\mathrm{v2}}^2 + {\mathrm{v3}}^2\right)}^{\frac{5}{4}}} - \frac{\mathrm{dv3}\, \mathrm{v2}\, \mathrm{v3}\, \sin\!\left(\sqrt{{\mathrm{v1}}^2 + {\mathrm{v2}}^2 + {\mathrm{v3}}^2}\right)}{{\left({\mathrm{v1}}^2 + {\mathrm{v2}}^2 + {\mathrm{v3}}^2\right)}^{\frac{5}{4}}}\\ \frac{\mathrm{dv3}\, \sin\!\left(\sqrt{{\mathrm{v1}}^2 + {\mathrm{v2}}^2 + {\mathrm{v3}}^2}\right)}{\sqrt{{\mathrm{v1}}^2 + {\mathrm{v2}}^2 + {\mathrm{v3}}^2}} + \frac{\mathrm{dv3}\, {\mathrm{v3}}^2}{{\left({\mathrm{v1}}^2 + {\mathrm{v2}}^2 + {\mathrm{v3}}^2\right)}^{\frac{3}{4}}} - \frac{\mathrm{dv3}\, {\mathrm{v3}}^2\, \sin\!\left(\sqrt{{\mathrm{v1}}^2 + {\mathrm{v2}}^2 + {\mathrm{v3}}^2}\right)}{{\left({\mathrm{v1}}^2 + {\mathrm{v2}}^2 + {\mathrm{v3}}^2\right)}^{\frac{5}{4}}} + \frac{2\, \mathrm{dv2}\, \mathrm{v1}\, {\sin\!\left(\frac{\sqrt{{\mathrm{v1}}^2 + {\mathrm{v2}}^2 + {\mathrm{v3}}^2}}{2}\right)}^2}{{\mathrm{v1}}^2 + {\mathrm{v2}}^2 + {\mathrm{v3}}^2} - \frac{2\, \mathrm{dv1}\, \mathrm{v2}\, {\sin\!\left(\frac{\sqrt{{\mathrm{v1}}^2 + {\mathrm{v2}}^2 + {\mathrm{v3}}^2}}{2}\right)}^2}{{\mathrm{v1}}^2 + {\mathrm{v2}}^2 + {\mathrm{v3}}^2} + \frac{\mathrm{dv1}\, \mathrm{v1}\, \mathrm{v3}}{{\left({\mathrm{v1}}^2 + {\mathrm{v2}}^2 + {\mathrm{v3}}^2\right)}^{\frac{3}{4}}} + \frac{\mathrm{dv2}\, \mathrm{v2}\, \mathrm{v3}}{{\left({\mathrm{v1}}^2 + {\mathrm{v2}}^2 + {\mathrm{v3}}^2\right)}^{\frac{3}{4}}} - \frac{\mathrm{dv1}\, \mathrm{v1}\, \mathrm{v3}\, \sin\!\left(\sqrt{{\mathrm{v1}}^2 + {\mathrm{v2}}^2 + {\mathrm{v3}}^2}\right)}{{\left({\mathrm{v1}}^2 + {\mathrm{v2}}^2 + {\mathrm{v3}}^2\right)}^{\frac{5}{4}}} - \frac{\mathrm{dv2}\, \mathrm{v2}\, \mathrm{v3}\, \sin\!\left(\sqrt{{\mathrm{v1}}^2 + {\mathrm{v2}}^2 + {\mathrm{v3}}^2}\right)}{{\left({\mathrm{v1}}^2 + {\mathrm{v2}}^2 + {\mathrm{v3}}^2\right)}^{\frac{5}{4}}} \end{array}\right)
\end{equation}

If one merely wishes to type in ordinary text, without
complicated mathematical formulae or special effects such
as font changes, then one merely has to type it in as it
is, leaving a completely blank line between successive
paragraphs.

You do not have to worry about paragraph indentation:
all paragraphs will be indented with the exception of
the first paragraph of a new section.

One must take care to distinguish between the `left quote'
and the `right quote' on the computer terminal.  Also, one
should use two `single quote' characters in succession if
one requires ``double quotes''.  One should never use the
(undirected) `double quote' character on the computer
terminal, since the computer is unable to tell whether it
is a `left quote' or a `right quote'.  One also has to
take care with dashes: a single dash is used for
hyphenation, whereas three dashes in succession are required
to produce a dash of the sort used for punctuation---such as
the one used in this sentence.

\section{Section Headings}

We explain in this section how to obtain headings
for the various sections and subsections of our
document.

\subsection{Headings in the `article' Document Style}

In the `article' style, the document may be divided up
into sections, subsections and subsubsections, and each
can be given a title, printed in a boldface font,
simply by issuing the appropriate command.

\end{document}
