\documentclass[a4paper,10pt]{article}
%\usepackage[landscape]{geometry}
\usepackage{color}
\usepackage{amsmath}
\usepackage{graphicx}

\definecolor{orange}{rgb}{1,0.5,0}
\definecolor{opaqueblue}{rgb}{0.5,0.5,1.0}
\begin{document}

%%%%%%%%% TITLE
\title{A biomechanical approach on biped dynamics}

\author{Geoyeob Kim\\
Computer Graphics Lab., KAIST
}
\date{2010}
\maketitle



\section{Introduction}
A character animation technique related to bipedal characters is
a very interesting research topic since it is an essential component
of every movie and video game. By designing and controlling the character
With a consistent dynamics model, a realistic and physically correct
motion can be synthesized automatically instead of key-framing the motion
manually. There has been a lot of approaches related to
synthesize the character motion with a physically-simulated manner.
Most of these work are based on a model with articulated figures and
actuated joints. In this model the character's body segments are considered
as a set of rigid bodies. Joints are connecting these rigid bodies.
This kind of model is
vastly used since it is suitable to formulate the Lagrangian equations of motion
using the generalized coordinates which helps to reduce the number
of degree-of-freedom (DoF).
The joints used in the model are considered as ideal rigid joints, which means
each of them connects the two rigid bodies with a certain DoF. For example,
a hinge joint which has 1-DoF is used at knees or elbows and a ball joint
which has 3-DoF is used at hips. Actuation on the joints are occurred
in their respective DoF. So it is natural to call these kind of joints
\emph{actuated motor joints}. Various animations can be created by
controlling these motor joints properly. In robotics this kind of
approach is much more
common. The main interest of creating motions with dynamics model
 is focused on how to design the motor joint controllers.


The motor joints are inherently nonsense in the perspective of
physical realism. No human being has that kind of motors in their joints.
Instead, they control their legs or arms by pulling bones using attached
muscles. Unfortunately, the whole mechanism is quite complicated to understand completely.
However the structure of human muscle fibers is well-known and there exist
the biomechanical experiments concerning the most basic human actions
such as walking and jumping.[REF NEEDED]

This thesis introduces a biomechanical approach on biped dynamics. Basically,
we introduces muscle fibers as the only actuating elements
of the biped and get rid of all rigid motor joints existed in the previous biped
model. Therefore joints are not rigid anymore. Instead, in the place of joints,
we use ligaments to hold the bodies firmly but not rigidly. This substitution
simulates human cartilages. The cartilages provide a shock absorption
system to sustain stability when external or internal disturbances exist.
We control the rigid bodies with muscle fibers which are modeled as
spring-damper system. Every muscle fiber has a linear actuator and
motions are created by controlling the actuator. Instead of controlling
the bodies with motor joints, it is highly effective in energy concern
since torques are created more easily by pulling the bones with the muscle
fiber on a distant place from an axis of rotation such as hips or ankles.

Human gait analysis is also very extensively studied area.[REF] It deals
with the mechanisms of human locomotion by throughly experimenting and measuring
the subjects. They classified human walking cycle in which a division
has its own objectives and features.
In computer animation, the most of work so far did not
exploit this kind of research results. They rely on optimization-based
techniques such as a motion tracker with a linear quadratic regulator (LQR)
or a quadratic programming optimizer attained by heuristic observations
such as 'minimize the actuation energy all the time' or 'follow the
reference trajectory with all your best'. They blindly seek the
natural human behavior of locomotion with these macroscopic objectives.
Although it seemed to have succeeded for generating nice-looking motion,
it can be enhanced by introducing biomechanical knowledge into the both of
a dynamics model and an optimization scheme.

The optimization-based approach is also used in this thesis. However,
we take gait phases analyzed in [PERRY] into account. Given motion
capture data containing locomotion of human, we classify the motion
sequence automatically into respective phases. For each phase,
an optimization formulation is built to calculate muscle actuation forces
of our biped model. The passive and active muscles are explicitly defined
with respect to a corresponding phase.
This helps our approach speed-up the simulation
and also gives more energy-efficient and biomechanically-correct result.
[CLEAR MENTION OF CONTRIBUTION IS NEEDED]


\section{Related work}
Designing and controlling a physically simulated character is a long-standing
challenge in robotics, biomechanics and computer graphics. A physically
synthesized motion that actively responds to the environment is quite
promising because it is hard to keyframing a natural animation for every motion
manually. The combination of motion capture sequence and manual keyframing
shows a quality of motion. However it is not straightforward to make a character 
interacts with its environment if a prerecorded motion is used.
By adapting physical simulation into a character animation we can overcome
this issue. The problem is that still there is no ultimate solution for controlling
a simulated character motion.

Muico and his colleagues presented a contact-aware controller. It is
heavily based on the theory of optimal control described in [LEWIS]
especially a linear quadratic regulator (LQR). They modified the LQR
to consider ground reaction forces. This controller is basically
a reference-tracking controller. This means that it is not applicable
to the environment which has large variations since that kind of environments
requires intentionally deviated reference trajectories.

KangKang Yin and her colleagues presented a simple proportional-derivative (PD) control framework.[SIMBICON]
Compared to the complexity of dynamics model, the structure of controller
is quite simple and robust to external disturbances. Since they do not
consider an energy efficiency or passive forces, the animation shows a kind of
stiff behavior.

A Optimization-based approach shown to be a good way to address many challenges
related to the physics-based character animation. Most of work advocate
their goal to determine the joint torques to make a certain desired pose.
A pose is calculated by integrating these torques.
Instead of finding torques, Jain and his colleagues formulated
an objective function which finds joint angles directly. The joint torques
are implicitly calculated in the optimization process. The advantage is
that it is easy to design various types of objectives such as place an arm here
or move a foot to this place since they used joint angles as optimization
variables.



Our goal is not so different from these work. However these work mainly focused
on the controller design and not the character model itself. Our work is
partly relies on the idea stuided in the field of biomechanics.
Some Researchers in biomechanical engineering focus on the movement
disorder and the biomedical applications such as artificial joints.
There are abundant work exist which use a model similar to
a real human body since it is essential to study the mechanism of underlying
structure of human body. Thelen and his colleagues illustrated
the algorithm to compute of muscle excitation patterns that produce
coordinated movements of muscle-actuated dynamic models. They simulated
a pedaling motion while fixing the upper body to investigate the excitation
pattern of 30 muscle fibers.

Liu and her colleagues tried to exploit the passive characteristics of
muscle fiber by including a torsion spring in joints. However the actuation
is still occurred by joint motors and the muscle fibers are not explicitly used.

The action for each muscle groups can be anticipated by adopting the
fact regarding to gait phases [PERRY]. The gait phases are the essential component
when analyzing a person's walking pattern. It directly identifies the
functional significance of the different motions occurring at the
individual joints. While controlling the character we
can determine the current gait phase easily. For each gait phase,
a set of muscle fibers from the entire muscles is selected to be actuated
and the others will be unactuated. This approach will give us an explicit
way to make full use of passive energy.

The muscle fiber model we use is composed of two springs and a damper.
This simple muscle model is explained in detail in the supplementary
documents [SHADMEHR]. In the document they assume that spring stiffness constants
and viscosity of the damper are constants. Some studies suggest that animals
vary stiffness according to the specific locomotion task.[FARLEY]
Since the model consists of springs we definitely need rest lengths for that
springs. However rest lengths are not simply defined on real muscles.
[SO WHAT?]

\section{Problem definition}

We model a character which consists of rigid bodies and virtual muscle fibers.
With this biomechanical model we design a reference trajectory tracker.
More specifically we will build the equations of motion which have the
form of
\begin{equation}\label{MoEq}
\mathbf{M}(\chi)\ddot\chi + C(\chi,\dot\chi) + G(\chi ) = f_c + \tau
\end{equation}
where $\mathbf{M}$ is a generalized mass matrix and $\chi$ is a generalized
linear and angular position
state vector and $\dot\chi$, $\ddot\chi$ are its first and second time derivatives,
respectively. $C$ represents Coriolis and centrifugal forces.
$G$ and $f_c$ represents gravitational forces and contact forces.
$\tau$ denotes muscle tension forces. For the case of a freely moving single
rigid body with 6-DoF, $\mathbf{M}$ will be $6\times 6$ matrix and $\chi$,
$f_c$, $\tau$, $C$ and $G$ will be a 6-dimensional vector. We have control
over none of the variables such as $\chi$, $\dot\chi$, $\ddot\chi$ and $f_c$.
Because $\chi$ and $\dot\chi$ are determined by integraing $\ddot\chi$ twice and once,
respectively. $\ddot\chi$ itself determined by external and inertial forces.
$f_c$ is also uncontrollable because it is the ground reaction force. This means that
$f_c$ is a reaction force arises by pushing the ground with the character's feet.
The only way to set $\chi$ to a certain desired value is controlling $\tau$
nicely. It is nontrivial to calculate such $\tau$ since \eqref{MoEq}
is an underactuated high dimensional nonlinear system.

There are two reasons why this problem is challenging. First it is difficult
to control dynamically simulated characters because they have no direct
control over their global position and orientation.

\section{Biped dynamics}

The concept of rigid body is commonly used in a physically-simulated
character model.
Our character model consists of 7 rigid bodies. One of these is
a head-trunk-arms (HAT) body segment. By using HAT instead of every upper body
segment, we can concentrate our concern to legs which are essential to
locomotion. The forward dynamics algorithm gives the following equation:


\begin{equation}
\ddot\chi =\mathbf{M}^{-1}(\chi) ( \tau - C(\chi,\dot\chi) ).
\end{equation}

We define a discrete-time state vector $Y$. For brevity, we use $Y^{(l)}$
to indicate a discrete-time value of $Y$ at $l$-th time step $Y(t_l)$.
We use an explicit Euler integration
scheme using both $Y$ and the equations of motion to derive a complete
forward dynamics formulation.

\begin{equation}
Y^{(l)} =
\left[ {\begin{array}{cc}
 \chi^{(l)}   \\
 \dot\chi^{(l)}   \\
 \end{array} } \right]
\end{equation}

\begin{equation}
f(Y^{(l)})=\dot{Y}^{(l)}
=
\left[ {\begin{array}{cc}
 \dot\chi^{(l)}   \\
 \ddot\chi^{(l)}   \\
 \end{array} } \right]
 =
\left[ {\begin{array}{cc}
 \dot\chi^{(l)}   \\
 \mathbf{M}^{-1}(\chi^{(l)}) ( \tau^{(l)} - C(\chi^{(l)},\dot\chi^{(l)}) )   \\
 \end{array} } \right]
\end{equation}

\begin{equation}
Y^{(l+1)}=Y^{(l)}+hf(Y^{(l)})
\end{equation}
Note that we need the position and velocity values of a certain time step
to store the state because the equations of motion for rigid bodies
are the second order differential equations.
If we model a dynamics system
which can be expressed by the first order differential equations of motion
then we need only a position as the state vector.

\subsection{Rigid body dynamics}

As stated in the previous section most of the existing work on
character animation used an articulated body to represent a character.
In the articulated body each body is connected to another body with
various types of joints. Since we use no joint at all in the character
model the mathematics involved in the dynamics system becomes quite
simple. We do not need to impose implicit or explicit constraints for
maintaining each body connected by joints stick together.
We assume that rigid bodies are independent to each other and thus we just build the
equations of motion which have $n$ rigid bodies floating around the
space freely with 6-DoF.

The equations of motion for a single rigid body is as follows:

\begin{equation}
\left[ {\begin{array}{cc}
 \bar{m}\mathbf{I}  &  0 \\
 0            & \mathbf{H}  \\
 \end{array} } \right]
 \left[ {\begin{array}{c}
 a  \\
 \alpha              \\
 \end{array} } \right]
 +
 \left[ {\begin{array}{c}
 0  \\
 \omega\times(\mathbf{H}\omega)   \\
 \end{array} } \right]
 =
 \left[ {\begin{array}{c}
 f  \\
 \tau   \\
 \end{array} } \right].
\end{equation}

This formulation is very easy to understand the structure of
the physical system. It reveals the linear and angular quantities
directly as well as the external forces and torques.


\begin{tabular}{ c c l }
Symbol       & Dimension           & Name \\
$\bar{m}$    &                     & body mass \\
$\mathbf{H}$ & $3 \times 3$        & moments of inertia \\
$a$          & $3 \times 1$        & linear acceleration \\ 
$\alpha$     & $3 \times 1$        & angular acceleration \\
$\omega$     & $3 \times 1$        & angular velocity \\
$f$          & $3 \times 1$        & external force\\
$\tau$       & $3 \times 1$        & external torque force\\
$n_d$        &                     & degree-of-freedom of a rigid body\\
$n$          &                     & the number of rigid bodies\\
$m$          &                     & the number of muscle fibers\\
$\chi$       & $n_d n  \times 1$   & generalized position\\
$\dot\chi$   & $n_d n  \times 1$   & generalized velocity\\
$\ddot\chi$  & $n_d n  \times 1$   & generalized acceleration\\
$\mathbf{M}$ & $n_d n  \times n_d$ & generalized mass matrix\\
$Y$          & $2n_d n \times 1$   & generalized state vector\\
$h$          &                     & simulation time step\\
$C$          & $2n_d n \times 1$   & Coriolis and centrifugal force\\
$k_{se}$     &                     & serial spring constant\\
$k_{pe}$     &                     & parallel spring constant\\
$x$          &                     & muscle fiber length\\
$x_r$        &                     & muscle fiber rest length\\
$b$          &                     & muscle fiber viscosity\\
$T$          &                     & muscle fiber tension\\
$A$          &                     & muscle fiber actutation force
\end{tabular}




\subsection{Muscle fiber model}

There are many variants of muscle model used in biomechanical area range
from a simple spring model to a time-varying nonlinear model. Although
a more sophisticated model gives us a more realistic behavior, unnecessarily
complicated models will make the biped hard to simulate and control. For a compromise,
we use one of the simplest time-invariant spring-damper model described in [SHADMEHR].
Basically, the model consists of two springs and a viscous damper.
The damper and the spring are connected in parallel and one spring is
connected in serial way. The active component in which our controller
can actuate resides in parallel side. A bundle of muscle fibers can be
categorized according to their structure and we treat this bundle as
a single fiber to simplify the model. For each muscle fiber, we need three
parameters: a serial spring constant $k_{se}$, a parallel spring constant $k_{pe}$,
a viscosity $b$ and a rest length $x_{r}$. The model can be formulated
as a first-order ordinary differential equation of a tension $T$ exerted
at the end of the fiber:

\begin{equation}\label{Tension}
\dot{T} = \frac{k_{se}}{b} \left( k_{pe}(x-x_{r})+b\dot{x}-\left(1+\frac{k_{pe}}{k_{se}}\right)T+A   \right)
\end{equation}
where $x$ is the length of the fiber and $A$ is applied force in the active
component. Note that every quantities shown in the equation are scalar values.
Since a muscle fiber always connects two bones (rigid bodies) and the
tension applied to the both of them, we need a clear definition of signed-ness
of the tension $T$. Researchers in biology defined the term $origin$ and $insertion$.
Origin and insertion indicate where a muscle fiber originated from and inserted
to, respectively. If we call the origin and insertion position of the fiber
as $p_{org}$ and $p_{ins}$, the normalized direction of the fiber can be
defined as $\hat{d}=(p_{ins}-p_{org})/||p_{ins}-p_{org}||$. With these quantities, we can think that
the tension $T\hat{d}$ and $-T\hat{d}$ are applied to the origin and insertion body
of the fiber, respectively.

THE INTERPRETATION OF THE EQUATION GOES HERE

\subsection{Whole biped model}

By modeling the bones as rigid bodies and the bunch of muscles as several muscle
fibers we can build the whole biped model. We have $n$ rigid bodies and
$m$ muscle fibers on the biped. The state vector $Y$ can be defined:

\begin{equation}
Y = [ p_1, q_1, \dot{p}_1, \dot{q}_1, \cdots ,
      p_n, q_n, \dot{p}_n, \dot{q}_n,
      T_1, \cdots, T_m]
\end{equation}
where $p_i$ and $q_i$ represent linear and angular position of body $i$.
Note that we omitted the transpose operator on vector quantities like $p$,
$q$ and $Y$ for brevity. A time derivative of $Y$ can be easily
calculated.

\begin{equation}
\dot{Y} = [ \dot{p}_1, \dot{q}_1, \ddot{p}_1, \ddot{q}_1, \cdots ,
            \dot{p}_n, \dot{q}_n, \ddot{p}_n, \ddot{q}_n,
            \dot{T}_1, \cdots, \dot{T}_m]
\end{equation}
Our concern is mostly concentrated to the quantities like $\ddot{p}$,
$\ddot{q}$ and $\dot{T}$ since the differential equations are formulated
in these variables. The vector $\dot{Y}$ is mixture
of the dynamic formula for rigid bodies and muscle fibers. We can separate
the vector into two parts to help the further analysis and control like the
following:

\begin{align}
\dot{Y} = {} & \quad[ \dot{p}_1, \dot{q}_1, \ddot{\tilde{p}}_1, \ddot{\tilde{q}}_1, \cdots ,
                \dot{p}_n, \dot{q}_n, \ddot{\tilde{p}}_n, \ddot{\tilde{q}}_n,
                0, \cdots, 0]\notag\\
             & + [ 0, 0, \ddot{\bar{p}}_1, \ddot{\bar{q}}_1, \cdots ,
                0, 0, \ddot{\bar{p}}_n, \ddot{\bar{q}}_n,
                \dot{T}_1, \cdots, \dot{T}_m]\notag\\
        = {} & \dot{Y}_R + \dot{Y}_Q\label{SeparatedYdot}
\end{align}

Since we separate the rigid body parts and muscle fiber parts, if we
assume $\dot{Y}_Q$ equals to 0 we simply simulate the $n$ individual
rigid bodies freely move around the space with rotations. If we develop
this kind of separation process further, we get the equations:

\begin{align}
\dot{Y}_R = {} &  \quad[ (\dot{p}_1, \dot{q}_1, \ddot{\tilde{p}}_1, \ddot{\tilde{q}}_1), 0_{4x}, (0, \cdots , 0)]\notag\\
               & +[ 0_{4x}, (\dot{p}_2, \dot{q}_2, \ddot{\tilde{p}}_2, \ddot{\tilde{q}}_2), 0_{4x}, (0, \cdots , 0)]\notag\\
               & + \cdots\notag\\
               & + [ 0_{4x}, (\dot{p}_n, \dot{q}_n, \ddot{\tilde{p}}_n, \ddot{\tilde{q}}_n), (0, \cdots , 0)]\notag\\
          = {} & \sum_{i=1}^{n}{\dot{Y}_{R,i}}
\end{align}

\begin{align}
\dot{Y}_Q = {} & \quad[ 0_{4x}, (0, 0, \ddot{\bar{p}}_{T,1}^{org}, \ddot{\bar{q}}_{T,1}^{org}),
                   0_{4x}, (0, 0, \ddot{\bar{p}}_{T,1}^{ins}, \ddot{\bar{q}}_{T,1}^{ins}),
                   0_{4x}, (\dot{T}_1, 0, \cdots, 0)]\notag\\
               & + [ 0_{4x}, (0, 0, \ddot{\bar{p}}_{T,2}^{ins}, \ddot{\bar{q}}_{T,2}^{ins}),
                     0_{4x}, (0, 0, \ddot{\bar{p}}_{T,2}^{org}, \ddot{\bar{q}}_{T,2}^{org}),
                     0_{4x}, (0, \dot{T}_2, 0, \cdots, 0)]\notag\\
               & + \cdots\notag\\
               & + [ 0_{4x}, (0, 0, \ddot{\bar{p}}_{T,m}^{org}, \ddot{\bar{q}}_{T,m}^{org}),
                     0_{4x}, (0, 0, \ddot{\bar{p}}_{T,m}^{ins}, \ddot{\bar{q}}_{T,m}^{ins}),
                     0_{4x}, (0, \cdots, 0, \dot{T}_m)]\notag\\
          = {} & \sum_{i=1}^{m}{\dot{Y}_{Q,i}}
\end{align}
Here $0_{4x}$ denotes an arbitrary zero vector whose dimension is multiple of four.
Parentheses are added just for clarify the structure.
$\dot{Y}_{R,i}$ represents the first derivative of the state vector of the body $i$
when there is no muscle fiber attached to it. $\dot{Y}_{Q,i}$ denotes the effects of
the muscle fiber $i$ affecting the origin and insertion body. Specifically,
$ \ddot{\bar{p}}_{T,i}^{org} $ and $ \ddot{\bar{q}}_{T,i}^{org} $ are the force and
torque affecting the origin body. This kind of separation gives us a simplified
way to analyze the whole system. For instance, if we set $\dot{Y}_{Q_i}$ to 0 then
the effect of muscle fiber $i$ is entirely removed from the simulation.

\subsection{Muscle fiber actuation force}
In \eqref{SeparatedYdot} we can see that the only way to affect the motion
of rigid bodies is changing $\dot{Y}_Q$. More specifically we need to change
$\dot{T}_i$ in $\dot{Y}_{Q,i}$. $\dot{T}$ is a linear function of the actuation force $A$
from \eqref{Tension} if we assume that $T$ is fixed or given at some time instant.
If we rewrite $\dot{T}$

\begin{align}
\dot{T} = {} &   \frac{k_{se}}{b} \left( k_{pe}(x-x_{r})+b\dot{x}-\left(1+\frac{k_{pe}}{k_{se}}\right)T \right)
               + \frac{k_{se}}{b}A\notag\\
        = {} & \dot{T}^U + \dot{T}^A\label{TensionLinear}
\end{align}
then $\dot{Y}_{Q,i}$ can be expressed in terms of $A$.
\begin{align}
\dot{Y}_{Q,i} = {} & [ \cdots , (0, 0, \ddot{\bar{p}}_{T,i}^{org}, \ddot{\bar{q}}_{T,i}^{org}),
                       \cdots , (0, 0, \ddot{\bar{p}}_{T,i}^{ins}, \ddot{\bar{q}}_{T,i}^{ins}),
                       \cdots , (\cdots , \dot{T}_i, \cdots )]\notag\\
              = {} & \quad[ \cdots , (0, 0, \ddot{\bar{p}}_{T,i}^{org}, \ddot{\bar{q}}_{T,i}^{org}),
                       \cdots , (0, 0, \ddot{\bar{p}}_{T,i}^{ins}, \ddot{\bar{q}}_{T,i}^{ins}),
                       \cdots , (\cdots , \dot{T}_i^U, \cdots )]\notag\\
                   &+[ \cdots , (0, 0, 0, 0),
                       \cdots , (0, 0, 0, 0),
                       \cdots , (\cdots , \dot{T}_i^A, \cdots )]\notag\\
              = {} & \dot{Y}_{Q,i}^U + \dot{Y}_{Q,i}^A\label{ActuationForceRevealed}
\end{align}
Because $\dot{Y}_{Q,i}^A$ depends on $A_i$ and $k_{se,i}$ and $b_i$ are constants
we can control this value arbitrarily as long as $A_i$ is in the range of
actuation force limit.
The muscle fiber $i$ will show a passive behavior of we set $A_i$ to zero.

\subsection{An example}

To explain the dynamics model in more specific manner, let us consider
a concrete example consists of three rigid bodies $R_1$, $R_2$ and $R_3$
and three muscle fibers $F_1$, $F_2$ and $F_3$. $R_1$ and $R_2$ are connected
by both of $F_1$ and $F_2$ whereas $F_3$ connects $R_2$ and $R_3$. The
origin and insertion of muscle fibers are described in Figure xx.

A 45-dimension state vector $Y$ is

\begin{align}
Y = {} & [p_1^x, p_1^y, p_1^z, q_1^w, q_1^x, q_1^y, q_1^z, \dot{p}_1^x, \dot{p}_1^y, \dot{p}_1^z, \dot{q}_1^w, \dot{q}_1^x, \dot{q}_1^y, \dot{q}_1^z,\notag\\
       &  p_2^x, p_2^y, p_2^z, q_2^w, q_2^x, q_2^y, q_2^z, \dot{p}_2^x, \dot{p}_2^y, \dot{p}_2^z, \dot{q}_2^w, \dot{q}_2^x, \dot{q}_2^y, \dot{q}_2^z,\notag\\
       &  p_3^x, p_3^y, p_3^z, q_3^w, q_3^x, q_3^y, q_3^z, \dot{p}_3^x, \dot{p}_3^y, \dot{p}_3^z, \dot{q}_3^w, \dot{q}_3^x, \dot{q}_3^y, \dot{q}_3^z,\notag\\
       & T_1, T_2, T_3]^T\label{StateVector}
\end{align}
where all variables are scalers. You may notice that we used quaternions
to parameterize the rotation of rigid bodies. This allows us a freedom
from the gimbal lock problem by sacrificing dimensional complexity. From
now on, we will denote the three or four dimensional vectors is its shorthand
form, i.e., $p_i=[p_i^x, p_i^y, p_i^z]^T$ and $q_i=[q_i^w, q_i^x, q_i^y, q_i^z]^T$.

\begin{equation}\label{StateVectorDerivative}
\dot{Y}  =  [\dot{p}_1, \dot{q}_1, \ddot{p}_1, \ddot{q}_1,
               \dot{p}_2, \dot{q}_2, \ddot{p}_2, \ddot{q}_2,
               \dot{p}_3, \dot{q}_3, \ddot{p}_3, \ddot{q}_3,
               \dot{T}_1, \dot{T}_2, \dot{T}_3]
\end{equation}
Again we omit the transpose operators at the places where they needed
unless the omission misleads our exposition.


\begin{align}
\dot{Y}_R = {} & \dot{Y}_{R,1} + \dot{Y}_{R,2} + \dot{Y}_{R,3}\\
          = {} &  [   \dot{p}_1, \dot{q}_1, \ddot{\tilde{p}}_1, \ddot{\tilde{q}}_1, 0, 0, 0, 0, 0, 0, 0, 0, 0, 0, 0 ]\notag\\
               & + [ 0, 0, 0, 0, \dot{p}_2, \dot{q}_2, \ddot{\tilde{p}}_2, \ddot{\tilde{q}}_2, 0, 0, 0, 0, 0, 0, 0 ]\notag\\
               & + [ 0, 0, 0, 0, 0, 0, 0, 0, \dot{p}_3, \dot{q}_3, \ddot{\tilde{p}}_3, \ddot{\tilde{q}}_3, 0, 0, 0 ]\notag
\end{align}


\begin{align}
\dot{Y}_Q = {} & \dot{Y}_{Q,1} + \dot{Y}_{Q,2} + \dot{Y}_{Q,3}\\
          = {} &   [ 0, 0, \ddot{\bar{p}}_{T,1}^{org}, \ddot{\bar{q}}_{T,1}^{org}, 0, 0, \ddot{\bar{p}}_{T,1}^{ins}, \ddot{\bar{q}}_{T,1}^{ins}, 0, 0, 0, 0, \dot{T}_1, 0, 0 ]\notag\\
               & + [ 0, 0, \ddot{\bar{p}}_{T,2}^{ins}, \ddot{\bar{q}}_{T,2}^{ins}, 0, 0, \ddot{\bar{p}}_{T,2}^{org}, \ddot{\bar{q}}_{T,2}^{org}, 0, 0, 0, 0, 0, \dot{T}_2, 0 ]\notag\\
               & + [ 0, 0, \ddot{\bar{p}}_{T,3}^{ins}, \ddot{\bar{q}}_{T,3}^{ins}, 0, 0, 0, 0, 0, 0, \ddot{\bar{p}}_{T,3}^{org}, \ddot{\bar{q}}_{T,3}^{org}, 0, 0, \dot{T}_3 ]\notag
\end{align}

\begin{equation}
\frac{\partial f}{\partial Y}
         =  \frac{\partial\dot{Y}_{R,1}}{\partial Y} + \frac{\partial\dot{Y}_{R,2}}{\partial Y} + \frac{\partial\dot{Y}_{R,3}}{\partial Y}
              + \frac{\partial\dot{Y}_{Q,1}}{\partial Y} + \frac{\partial\dot{Y}_{Q,2}}{\partial Y} + \frac{\partial\dot{Y}_{Q,3}}{\partial Y}
\end{equation}


\begin{equation}
y_{i} = [p_i, q_i, \dot{p}_i, \dot{q}_i]
\end{equation}

\begin{equation}
\dot{y}_{R,i} = [\dot{p}_i, \dot{q}_i, \ddot{\tilde{p}}_i, \ddot{\tilde{q}}_i]
\end{equation}

% User-defined macro for printing zero matrix in opaque color.
% Should be used in math blocks.
% Usage) \zm{2,3}
%        \zm{}
\newcommand{\zm}[1]{\ensuremath{ {\color{opaqueblue} 0_{#1} } }}

\begin{equation}
\frac{\partial\dot{Y}_{R,1}}{\partial Y}=
\left[ \begin{array}{cccc}
\partial \dot{y}_{R,1} / \partial y_1 & \zm{14,14} & \zm{14,14} & \zm{14,3}\\
\zm{14,14}                            & \zm{14,14} & \zm{14,14} & \zm{14,3}\\
\zm{14,14}                            & \zm{14,14} & \zm{14,14} & \zm{14,3}\\
\zm{3,14}                             & \zm{3,14}  & \zm{3,14}  & \zm{3,3}\\
\end{array} \right]
\end{equation}

\begin{equation}
\frac{\partial\dot{Y}_{R,2}}{\partial Y}=
\left[ \begin{array}{cccc}
\zm{14,14} & \zm{14,14}                            & \zm{14,14} & \zm{14,3}\\
\zm{14,14} & \partial \dot{y}_{R,2} / \partial y_2 & \zm{14,14} & \zm{14,3}\\
\zm{14,14} & \zm{14,14}                            & \zm{14,14} & \zm{14,3}\\
\zm{3,14}  & \zm{3,14}                             & \zm{3,14}  & \zm{3,3}\\
\end{array}  \right]
\end{equation}

\begin{equation}
\frac{\partial\dot{Y}_{R,3}}{\partial Y}=
\left[ \begin{array}{cccc}
\zm{14,14} & \zm{14,14} & \zm{14,14}                            & \zm{14,3}\\
\zm{14,14} & \zm{14,14} & \zm{14,14}                            & \zm{14,3}\\
\zm{14,14} & \zm{14,14} & \partial \dot{y}_{R,3} / \partial y_3 & \zm{14,3}\\
\zm{3,14}  & \zm{3,14}  & \zm{3,14}                             & \zm{3,3}\\
\end{array}  \right]
\end{equation}


\begin{equation}
\dot{y}_{Q,i}^{org} = [0, 0, \ddot{\bar{p}}_{T,i}^{org}, \ddot{\bar{q}}_{T,i}^{org}]
\end{equation}

\begin{equation}
\dot{y}_{Q,i}^{ins} = [0, 0, \ddot{\bar{p}}_{T,i}^{ins}, \ddot{\bar{q}}_{T,i}^{ins}]
\end{equation}

\begin{equation}
\frac{\partial\dot{Y}_{Q,1}}{\partial Y}=
\left[ \begin{array}{cccccc}
\partial\dot{y}_{Q,1}^{org} / \partial y_1 & \partial\dot{y}_{Q,1}^{org} / \partial y_2 & \zm{14,14} & \partial\dot{y}_{Q,1}^{org} / \partial T_1 & \zm{14,1} & \zm{14,1} \\
\partial\dot{y}_{Q,1}^{ins} / \partial y_1 & \partial\dot{y}_{Q,1}^{ins} / \partial y_2 & \zm{14,14} & \partial\dot{y}_{Q,1}^{ins} / \partial T_1 & \zm{14,1} & \zm{14,1} \\
\zm{14,14}                                 & \zm{14,14}                                 & \zm{14,14} & \zm{14,1}                                  & \zm{14,1} & \zm{14,1} \\
\partial\dot{T}_1 / \partial y_1           & \partial\dot{T}_1 / \partial y_2           & \zm{1,14}  & \partial\dot{T}_1 / \partial T_1           & \zm{}     & \zm{} \\
\zm{1,14}                                  & \zm{1,14}                                  & \zm{1,14}  & \zm{}                                      & \zm{}     & \zm{} \\
\zm{1,14}                                  & \zm{1,14}                                  & \zm{1,14}  & \zm{}                                      & \zm{}     & \zm{} \\
\end{array}  \right]
\end{equation}

\begin{equation}
\frac{\partial\dot{Y}_{Q,2}}{\partial Y}=
\left[ \begin{array}{cccccc}
\partial\dot{y}_{Q,2}^{ins} / \partial y_1 & \partial\dot{y}_{Q,2}^{ins} / \partial y_2 & \zm{14,14} & \zm{14,1} & \partial\dot{y}_{Q,2}^{ins} / \partial T_2 & \zm{14,1} \\
\partial\dot{y}_{Q,2}^{org} / \partial y_1 & \partial\dot{y}_{Q,2}^{org} / \partial y_2 & \zm{14,14} & \zm{14,1} & \partial\dot{y}_{Q,2}^{org} / \partial T_2 & \zm{14,1} \\
\zm{14,14}                                 & \zm{14,14}                                 & \zm{14,14} & \zm{14,1} & \zm{14,1}                                  & \zm{14,1} \\
\zm{1,14}                                  & \zm{1,14}                                  & \zm{1,14}  & \zm{}     & \zm{}                                      & \zm{} \\
\partial\dot{T}_2 / \partial y_1           & \partial\dot{T}_2 / \partial y_2           & \zm{1,14}  & \zm{}     & \partial\dot{T}_2 / \partial T_2           & \zm{} \\
\zm{1,14}                                  & \zm{1,14}                                  & \zm{1,14}  & \zm{}     & \zm{}                                      & \zm{} \\
\end{array}  \right]
\end{equation}

\begin{equation}
\frac{\partial\dot{Y}_{Q,3}}{\partial Y}=
\left[ \begin{array}{cccccc}
\partial\dot{y}_{Q,3}^{ins} / \partial y_1 & \zm{14,14} & \partial\dot{y}_{Q,3}^{ins} / \partial y_3 & \zm{14,1} & \zm{14,1} & \partial\dot{y}_{Q,3}^{ins} / \partial T_3  \\
\zm{14,14}                                 & \zm{14,14} & \zm{14,14}                                 & \zm{14,1} & \zm{14,1} & \zm{14,1} \\
\partial\dot{y}_{Q,3}^{org} / \partial y_1 & \zm{14,14} & \partial\dot{y}_{Q,3}^{org} / \partial y_3 & \zm{14,1} & \zm{14,1} & \partial\dot{y}_{Q,3}^{org} / \partial T_3  \\
\zm{1,14}                                  & \zm{1,14}  & \zm{1,14}                                  & \zm{}     & \zm{}     & \zm{} \\
\zm{1,14}                                  & \zm{1,14}  & \zm{1,14}                                  & \zm{}     & \zm{}     & \zm{} \\
\partial\dot{T}_3 / \partial y_1           & \zm{1,14}  & \partial\dot{T}_3 / \partial y_3           & \zm{}     & \zm{}     & \partial\dot{T}_3 / \partial T_3 \\
\end{array}  \right]
\end{equation}



\section{Contact model}
There are various kind of techniques to deal with the calculation of
contact forces between rigid bodies. The existing algorithms can be
divided into three categories: constraint-based, penalty-based and
impulse-based approaches. In constraint-based methods, the contact
forces are computed by solving an optimization problem such as
linear complementarity problems (LCP). The contact forces are calculated
precisely in this case. It is not well suited in the environments
which contain many contact points. Penalty-based methods compute
the contact forces by considering a contact as a spring-damper.
It is scalable and easy to implement. However it requires very short
integration step especially at collisions and in most cases nonpenetration
conditions are violated. Impulse-based methods model contacts
as successive collisions instead of persisting contacts.

The simulation environment contains a relatively small number
of contact points. The precise contact forces are necessary because
the global alignment of the character only depends on that forces.
Although physically conforming contact forces are not strictly needed in
the point of believable character animation, our controller uses the
ground reaction forces as input to compute desired actuation forces.

We choose a time-stepping method introduced in [STEWART, ANITESCU].
If we form a LCP by using a mass matrix and a Coriolis and centrifugal force vector
derived in a generalized coordinate as well as a state and its time derivatives
the solution will give us contact forces at the instant.


\section{Implicit integration}

Most of rigid body systems can be simulated using an explicit Euler method
as far as there is no stiff component exists. The explicit integration
allows us a simple implementation and fast computation of a next state,
however, it has the potential of instability when the time step is larger
then a certain threshold. The threshold is mainly determined by a time
constant of a given differential equation we need to integrate. In our case, the time
constant is dominated by muscle fibers because they consists
of highly stiff springs \eqref{Tension}. Even if we use very small time step
such as $h=10^{-6}$, our system diverges very quickly with the explicit
integration. To make the simulation stable, we used an implicit integration
method.

\begin{equation}
Y^{(l+1)}=Y^{(l)}+hf(Y^{(l+1)})
\end{equation}
That is, we want to find the next state $Y^{(l+1)}$ where we
reverse the time step by $-h$ from $Y^{(l+1)}$, we get to the current state
$Y^{(l)}$. In most cases, $f$ is a nonlinear function of $Y$, so we need to
linearize it to find closed form solution of $Y^{(l+1)}$. If we define a
variable $\Delta Y = Y^{(l+1)}-Y^{(l)}$, it can be calculated using the
following equation:

\begin{equation}\label{DeltaY}
\Delta Y = \left(  \frac{1}{h}\mathbf{1} - {\frac{\partial f}{\partial Y} \bigg|_{Y=Y^{(l)}}}\right)^{-1} f(Y^{(l)})
\end{equation}
where $\mathbf{1}$ is an identity matrix. We need an analytic form of
Jacobian $\partial f / \partial Y$ to calculate $\Delta Y$.
From \eqref{SeparatedYdot}, we can see that
Jacobian also can be calculated in a separated manner.


\begin{align}
\frac{\partial f}{\partial Y}
        & = \frac{\partial\dot{Y}}{\partial Y}\notag\\
        & = \frac{\partial\dot{Y}_R}{\partial Y} + \frac{\partial\dot{Y}_Q}{\partial Y}\notag\\
        & = \sum_{i=1}^{n}\frac{\partial\dot{Y}_{R,i}}{\partial Y} + \sum_{i=1}^{m}\frac{\partial\dot{Y}_{Q,i}}{\partial Y}
\end{align}

\includegraphics[width=120mm]{testplot}


\section{Gait phases}

Locomotion such as walking or running is performed by a periodic way.
A gait cycle or stride contains eight functional patterns(FIGURE xx).
[PERRY] These patterns are called \emph{gait phases}. Each phase has its
own features and objectives. For example, during mid-swing and terminal swing
phase, one of primary muscle called \emph{biceps femoris} and \emph{semimembranosis} is
activated to extend our swing leg forward.

terminal stance phase one
of the muscle  we will use this fact
to our locomotion controller.

Aha!
Change this...
fdsfsd
Whoooooooooooa~


\section{Control}

If we substitute \eqref{SeparatedYdot} and \eqref{ActuationForceRevealed} into
\eqref{DeltaY} the control structure of $\Delta Y$ is shown.

\begin{equation}\label{DeltaYControl}
\Delta Y = \left(  \frac{1}{h}\mathbf{1} - {\frac{\partial f}{\partial Y} \bigg|_{Y=Y^{(l)}}}\right)^{-1} 
            \left( \dot{Y}_R^{(l)} + \dot{Y}_Q^{U(l)} + \dot{Y}_Q^{A(l)} \right)
\end{equation}
$\Delta Y$ can be rearranged to be affine in the control argument $u$ as follows:

\begin{equation}
\Delta Y(Y^{(l)}, u) = C(Y^{(l)}) + \mathbf{D}(Y^{(l)})u
\end{equation}
where
\begin{equation}
C(Y^{(l)}) = \mathbf{P}^{-1}(Y^{(l)}) \left( \dot{Y}_R^{(l)} + \dot{Y}_Q^{U(l)} \right),
\end{equation}

\begin{equation}
\mathbf{D}(Y^{(l)}) = \mathbf{P}^{-1}(Y^{(l)}) \mathbf{G},
\end{equation}

\begin{equation}\label{DeltaYMatrix}
\mathbf{P}(Y^{(l)}) = \frac{1}{h}\mathbf{1} - {\frac{\partial f}{\partial Y} \bigg|_{Y=Y^{(l)}}},
\end{equation}

\begin{equation}
\mathbf{G} = \left[ \zm{m, 2n_d n} , \text{diag}(k_{se,1}/b_1, \cdots, k_{se,m}/b_m) \right]^T,
\end{equation}

\begin{equation}
u = [A_1, \cdots, A_m]^T.
\end{equation}

The controller is based on an optimization. Assume that we are in
the time step $l$ with the state $Y^{(l)}$ and we want to determine $u$
which leads the next time state $Y^{(l+1)}$ to $Y_{\text{desired}}$
as close as possible while using minimal efforts to achieve that goal.
The optimization problem for finding the solution vector $u^*$ can be stated as

\begin{align}
u^*= {} & \arg\min_{u} \| Y^{(l+1)}                            - Y_\text{desired} \|^2_{\mathbf{W}_Y} + \| \mathbf{F}u \|^2_{\mathbf{W}_u} \notag\\
   = {} & \arg\min_{u} \| (Y^{(l)} + \Delta Y)                 - Y_\text{desired} \|^2_{\mathbf{W}_Y} + \| \mathbf{F}u \|^2_{\mathbf{W}_u}\notag\\
   = {} & \arg\min_{u} \| (Y^{(l)} + C + \mathbf{D}u) - Y_\text{desired} \|^2_{\mathbf{W}_Y} + \| \mathbf{F}u \|^2_{\mathbf{W}_u}\notag\\
   = {} & \arg\min_{u} \| \mathbf{D}u + E \|^2_{\mathbf{W}_Y}  + \| \mathbf{F}u \|^2_{\mathbf{W}_u}
\end{align}
where
\begin{equation}
E = Y^{(l)} - Y_\text{desired} + C ,
\end{equation}
\begin{equation}
\mathbf{F} = \left[ \begin{array}{c}
\zm{2n_d n,m}\\
\mathbf{1}_m
\end{array}  \right].
\end{equation}
The weighting between different objective terms is
usually determined by diagonal semipositive definite matrices $\mathbf{W}_Y$ and $\mathbf{W}_u$.
Note that $\mathbf{D}$ can be a rectangular (nonsquare) matrix depending on the number of rigid bodies and
muscles. The equations for finding $u^*$ is as follows:
\begin{align}
\frac{\partial}{\partial u} \left( \| \mathbf{D}u + E \|^2_{\mathbf{W}_Y}  + \| \mathbf{F}u \|^2_{\mathbf{W}_u} \right) = {} & 0  \notag\\
2\mathbf{D}^T\mathbf{W}_Y(\mathbf{D}u+E) + 2\mathbf{F}^T\mathbf{W}_u \mathbf{F}u = {} & 0\notag\\
(\mathbf{D}^T\mathbf{W}_Y\mathbf{D}+\mathbf{F}^T\mathbf{W}_u\mathbf{F})u = {} & -\mathbf{D}^T\mathbf{W}_Y E
\end{align}
Here we define the gradient as a column vector.

\section{Putting it all together}

\section{Result}

\section{Conclusion}


\end{document}
