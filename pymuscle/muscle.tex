\documentclass[a4paper,10pt]{article}
%\usepackage[landscape]{geometry}
\begin{document}


%%%%%%%%% TITLE
\title{A biomechanical approach on biped dynamics}

\author{Geoyeob Kim\\
Computer Graphics Lab., KAIST
}
\date{2010}
\maketitle

\section{Introduction}
Biped character animation technique is a very interesting
research topic since it is an essential component of almost every movies
and video game industries. By designing and controlling the character
with a consistent dynamics model, a realistic and physically correct
motion can be synthesized automatically instead of manually keyframing
the motion. There has been a lot of approaches related to
synthesize the character motion with a physically-simulated manner.
Most of these work are based on a model with articulated figures and
actuacted joint motors. In this model, human's body parts are considered
as rigid bodies. Joints are connecting these rigid bodies. This kind of model is
vastly used since it is suitable to formulate the Lagrangian equation of motion
using the generalized coordinates which helps to reduce the number of degree of freedom(DoF).
The joints used in the model are ideal rigid joints, which means
each of them connects the two rigid bodies with a certain DoF. For example,
a hinge joint which has 1-DoF is used at knees or elbows and a ball joint
which has 3-DoF is used at hips. Actuation on the joints are occured
in their respecive DoFs. So it is natural to call these kind of joints
\emph{motor joints}. Various animations can be created by controlling these joint
motors properly. In robotics area, this kind of approach is much more
common. So far the heart of creating the motion with dynamics model is
how to design motor controllers.

However, the motor joints are inherently nonsense in the perspective of
physical realism. No human has that kind of motors in their joints.
Instead, they control their legs or arms by pulling bones using attached
muscle fibers. This mechanism is quite complicated to understand completely.
However, the structure of human muscle fibers is well-known and the biomechanical
experiments concerning the most basic actions like walking are exist.

This thesis introduces a biomechanical approach on biped dynamics. Basically,
we introduces muscle fibers as the only actuating elements
of the biped and get rids of all motor joints existed in typical biped
model. Joints are not rigid anymore. Instead, in the place of joints,
we use ligaments to hold the bodies firmly but not rigidly to simulate
cartilages. The ligaments connecting rigid bodies provide a shock absorption
structure to sustain stability when external or internal disturbances exist.
Besides, we control the rigid bodies with muscle fibers which are modeled as
spring-damper system. Every muscle fiber has a linear actuator and the
motions are created by controlling these actuators. Instead of controlling
the bodies with motor joints, it is highly effective in energy concern
since torques are created more easily by pulling the bones with the muscle fiber.

Gait analysis is also very extensively studied area. It concerns the
mechanisms of human locomotion by throughly experementing and measuring
the subjects. They classified the human walking cycle into some divisions
and each division has its own objectvies. The most of work so far did not
exploit this kind of research results. They rely on optimization-based
techniques such as a motion tracker with linear quadratic regulator (LQR)
or a optimizer with heuristic objective functions not concerning the
biological human behavior.

The optimization-based approach is also used in this thesis. However,
we take gait phases analyzed in [Perry] into account. Given motion
capture data containing locomotion of human, we classify the motion
sequence automatically into respective phases. For each phase,
an optimization formulation is built to calculate muscle actuation forces
of our biped model. The passive and active muscles are explicitly defined
with respect to a phase. This helps our approach speed-up the simulation
and also gives more energy-efficient and biologically-correct result.

Since the goal of motion synthesizing of humanoid characters is to
create desired and realistic motion without laborious keyframing, we
need not only a proper controlling scheme but also 



\section{Related work}

\section{Problem definition}

\section{Biped dynamics}

Rigid body dynamics are commonly used in physical character model.
Our character model consists of 7 rigid bodies. One of these is
head-trunk-arms (HAT) body. By using HAT instead of every upper body
part, we can concentrate our concern to legs which are essential to
walking. The forward dynamics algorithm gives the following equation:


\begin{equation}
\ddot\chi =\mathbf{M}^{-1}(\chi) ( \tau - C(\chi,\dot\chi) )
\end{equation}
where $\mathbf{M}$ is a mass matrix and $\chi$ is a linear and angular position
vector and $\dot\chi$, $\ddot\chi$ are its first and second time derivatives,
respectively. $C$ represents all the other forces except the actuated and
external forces represented as $\tau$. For the case of freely moving single
rigid body with 6-DoF, $\mathbf{M}$ will be $6\times 6$ matrix and $\chi$,
$\tau$ and $C$ will be a 6-dimensional vector.

We define a discrete-time state vector $Y$. For brevity, we use $Y^{(l)}$
to indicate a discrete-time value of $Y$ at $l$-th time step $Y(t_l)$.
We use an explicit Euler integration
scheme using both $Y$ and the equations of motion to yield a complete
forward dynamics formulation.


\begin{equation}
Y^{(l)} =
\left[ {\begin{array}{cc}
 \chi^{(l)}   \\
 \dot\chi^{(l)}   \\
 \end{array} } \right]
\end{equation}


\begin{equation}
f(Y^{(l)})=\dot{Y}^{(l)}
=
\left[ {\begin{array}{cc}
 \dot\chi^{(l)}   \\
 \ddot\chi^{(l)}   \\
 \end{array} } \right]
 =
\left[ {\begin{array}{cc}
 \dot\chi^{(l)}   \\
 \mathbf{M}^{-1}(\chi^{(l)}) ( \tau^{(l)} - C(\chi^{(l)},\dot\chi^{(l)}) )   \\
 \end{array} } \right]
\end{equation}

\begin{equation}
Y^{(l+1)}=Y^{(l)}+hf(Y^{(l)})
\end{equation}

\subsection{Rigid body dynamics}

\subsection{Muscle fiber model}

There are many variants of muscle model used in biomechanical area range
from a simple spring model to a time-varying nonlinear model. Although
a more sophisticated model gives us a more realistic behavior, unnecessarily
complicated models will make the biped hard to simulate. For a compromise,
we use one of the simplest time-invariant spring-damper model described in [Shadmehr].
Basically, the model consists of two springs and a viscous damper.
The damper and the spring are connected in parallel and one spring is
connected in serial way. The active component in which our controller
can actuate resides in parallel side. A bundle of muscle fibers can be
categorized according to their structure and we treat this bundle as
a single fiber to simplify the model. For each muscle fiber, we need three
parameters: a serial spring constant $k_{se}$, a parallel spring constant $k_{pe}$,
a viscosity $b$ and a rest length $x_{r}$. The model can be formulated
as a first-order ordinary differential equation of a tension $T$ exerted
at the end of the fiber:

\begin{equation}
\dot{T} = \frac{k_{se}}{b} \left( k_{pe}(x-x_{r})+b\dot{x}-\left(1+\frac{k_{pe}}{k_{se}}\right)T+A   \right)
\end{equation}
where $x$ is the length of the fiber and $A$ is applied force in the active
component. Note that every quantities shown in the equation are scalar values.
Since a muscle fiber always connects two bones (rigid bodies) and the
tension applied to the both of them, we need a clear definition of signedness
of the tension $T$. Researchers in biology defined the term $origin$ and $insertion$.
Origin and insertion indicate where a muscle fiber originated from and inserted
to, repectively. If we call the origin and insertion position of the fiber
as $p_{org}$ and $p_{ins}$, the normalized direction of the fiber can be
defined as $\hat{d}=(p_{ins}-p_{org})/||p_{ins}-p_{org}||$. With these quantities, we can think that
the tension $T\hat{d}$ and $-T\hat{d}$ are applied to the origin and insertion body
of the fiber, respectively.

THE INTERPRETATION OF THE EQUATION GOES HERE

\subsection{Whole biped model}

By modeling the bones as rigid bodies and the bunch of muscles as muscle
fibers described so far, we can build the whole biped model.


\section{Contact model}

\section{Implicit integration}

\section{Gait phases}

\section{Controller design}


\section{Result}

\section{Conclusion}


\begin{equation}
\mathbf{A}\vec{\ddot{f}}_b(x) + g(x) = 4x + 11dd
\end{equation}

\begin{eqnarray*}
m\vec{a} & = & \vec{f} \\
         & = & \vec{f}_c+\vec{f}_m+\vec{f}_e \\
         & = & \sum_{i=1}^{n_c}{\vec{f}_{c,i}} + \sum_{i=1}^{n_m}{\vec{f}_{m,i}} + \sum_{i=1}^{n_e}{\vec{f}_{e,i}}. \\
\\
\mathbf{H}\vec\alpha + \vec\omega\times\mathbf{H}\vec\omega
         & = & \vec\tau \\
         & = & \sum_{i=1}^{n_c}{\vec{r}_{c,i}\times\vec{f}_{c,i}} + \sum_{i=1}^{n_m}{\vec{r}_{m,i}\times\vec{f}_{m,i}} + \sum_{i=1}^{n_e}{\vec{r}_{e,i}\times\vec{f}_{e,i}}. \\
\end{eqnarray*}

\begin{equation}
\frac{\mathrm{v1}\, \cos\!\left(\frac{\sqrt{\mathrm{thetasq}}}{2}\right)}{2\, \mathrm{thetasq}} - \frac{\mathrm{v1}\, \sin\!\left(\frac{\sqrt{\mathrm{thetasq}}}{2}\right)}{{\mathrm{thetasq}}^{\frac{3}{2}}} - \frac{3\, \mathrm{v1}\, {\mathrm{v2}}^2\, \cos\!\left(\frac{\sqrt{\mathrm{thetasq}}}{2}\right)}{2\, {\mathrm{thetasq}}^2} - \frac{\mathrm{v1}\, {\mathrm{v2}}^2\, \sin\!\left(\frac{\sqrt{\mathrm{thetasq}}}{2}\right)}{4\, {\mathrm{thetasq}}^{\frac{3}{2}}} + \frac{3\, \mathrm{v1}\, {\mathrm{v2}}^2\, \sin\!\left(\frac{\sqrt{\mathrm{thetasq}}}{2}\right)}{{\mathrm{thetasq}}^{\frac{5}{2}}}
\end{equation}

\begin{equation}
\left(\begin{array}{c} \frac{\mathrm{dv1}\, \sin\!\left(\sqrt{{\mathrm{v1}}^2 + {\mathrm{v2}}^2 + {\mathrm{v3}}^2}\right)}{\sqrt{{\mathrm{v1}}^2 + {\mathrm{v2}}^2 + {\mathrm{v3}}^2}} + \frac{\mathrm{dv1}\, {\mathrm{v1}}^2}{{\left({\mathrm{v1}}^2 + {\mathrm{v2}}^2 + {\mathrm{v3}}^2\right)}^{\frac{3}{4}}} - \frac{\mathrm{dv1}\, {\mathrm{v1}}^2\, \sin\!\left(\sqrt{{\mathrm{v1}}^2 + {\mathrm{v2}}^2 + {\mathrm{v3}}^2}\right)}{{\left({\mathrm{v1}}^2 + {\mathrm{v2}}^2 + {\mathrm{v3}}^2\right)}^{\frac{5}{4}}} + \frac{2\, \mathrm{dv3}\, \mathrm{v2}\, {\sin\!\left(\frac{\sqrt{{\mathrm{v1}}^2 + {\mathrm{v2}}^2 + {\mathrm{v3}}^2}}{2}\right)}^2}{{\mathrm{v1}}^2 + {\mathrm{v2}}^2 + {\mathrm{v3}}^2} - \frac{2\, \mathrm{dv2}\, \mathrm{v3}\, {\sin\!\left(\frac{\sqrt{{\mathrm{v1}}^2 + {\mathrm{v2}}^2 + {\mathrm{v3}}^2}}{2}\right)}^2}{{\mathrm{v1}}^2 + {\mathrm{v2}}^2 + {\mathrm{v3}}^2} + \frac{\mathrm{dv2}\, \mathrm{v1}\, \mathrm{v2}}{{\left({\mathrm{v1}}^2 + {\mathrm{v2}}^2 + {\mathrm{v3}}^2\right)}^{\frac{3}{4}}} + \frac{\mathrm{dv3}\, \mathrm{v1}\, \mathrm{v3}}{{\left({\mathrm{v1}}^2 + {\mathrm{v2}}^2 + {\mathrm{v3}}^2\right)}^{\frac{3}{4}}} - \frac{\mathrm{dv2}\, \mathrm{v1}\, \mathrm{v2}\, \sin\!\left(\sqrt{{\mathrm{v1}}^2 + {\mathrm{v2}}^2 + {\mathrm{v3}}^2}\right)}{{\left({\mathrm{v1}}^2 + {\mathrm{v2}}^2 + {\mathrm{v3}}^2\right)}^{\frac{5}{4}}} - \frac{\mathrm{dv3}\, \mathrm{v1}\, \mathrm{v3}\, \sin\!\left(\sqrt{{\mathrm{v1}}^2 + {\mathrm{v2}}^2 + {\mathrm{v3}}^2}\right)}{{\left({\mathrm{v1}}^2 + {\mathrm{v2}}^2 + {\mathrm{v3}}^2\right)}^{\frac{5}{4}}}\\ \frac{\mathrm{dv2}\, \sin\!\left(\sqrt{{\mathrm{v1}}^2 + {\mathrm{v2}}^2 + {\mathrm{v3}}^2}\right)}{\sqrt{{\mathrm{v1}}^2 + {\mathrm{v2}}^2 + {\mathrm{v3}}^2}} + \frac{\mathrm{dv2}\, {\mathrm{v2}}^2}{{\left({\mathrm{v1}}^2 + {\mathrm{v2}}^2 + {\mathrm{v3}}^2\right)}^{\frac{3}{4}}} - \frac{\mathrm{dv2}\, {\mathrm{v2}}^2\, \sin\!\left(\sqrt{{\mathrm{v1}}^2 + {\mathrm{v2}}^2 + {\mathrm{v3}}^2}\right)}{{\left({\mathrm{v1}}^2 + {\mathrm{v2}}^2 + {\mathrm{v3}}^2\right)}^{\frac{5}{4}}} - \frac{2\, \mathrm{dv3}\, \mathrm{v1}\, {\sin\!\left(\frac{\sqrt{{\mathrm{v1}}^2 + {\mathrm{v2}}^2 + {\mathrm{v3}}^2}}{2}\right)}^2}{{\mathrm{v1}}^2 + {\mathrm{v2}}^2 + {\mathrm{v3}}^2} + \frac{2\, \mathrm{dv1}\, \mathrm{v3}\, {\sin\!\left(\frac{\sqrt{{\mathrm{v1}}^2 + {\mathrm{v2}}^2 + {\mathrm{v3}}^2}}{2}\right)}^2}{{\mathrm{v1}}^2 + {\mathrm{v2}}^2 + {\mathrm{v3}}^2} + \frac{\mathrm{dv1}\, \mathrm{v1}\, \mathrm{v2}}{{\left({\mathrm{v1}}^2 + {\mathrm{v2}}^2 + {\mathrm{v3}}^2\right)}^{\frac{3}{4}}} + \frac{\mathrm{dv3}\, \mathrm{v2}\, \mathrm{v3}}{{\left({\mathrm{v1}}^2 + {\mathrm{v2}}^2 + {\mathrm{v3}}^2\right)}^{\frac{3}{4}}} - \frac{\mathrm{dv1}\, \mathrm{v1}\, \mathrm{v2}\, \sin\!\left(\sqrt{{\mathrm{v1}}^2 + {\mathrm{v2}}^2 + {\mathrm{v3}}^2}\right)}{{\left({\mathrm{v1}}^2 + {\mathrm{v2}}^2 + {\mathrm{v3}}^2\right)}^{\frac{5}{4}}} - \frac{\mathrm{dv3}\, \mathrm{v2}\, \mathrm{v3}\, \sin\!\left(\sqrt{{\mathrm{v1}}^2 + {\mathrm{v2}}^2 + {\mathrm{v3}}^2}\right)}{{\left({\mathrm{v1}}^2 + {\mathrm{v2}}^2 + {\mathrm{v3}}^2\right)}^{\frac{5}{4}}}\\ \frac{\mathrm{dv3}\, \sin\!\left(\sqrt{{\mathrm{v1}}^2 + {\mathrm{v2}}^2 + {\mathrm{v3}}^2}\right)}{\sqrt{{\mathrm{v1}}^2 + {\mathrm{v2}}^2 + {\mathrm{v3}}^2}} + \frac{\mathrm{dv3}\, {\mathrm{v3}}^2}{{\left({\mathrm{v1}}^2 + {\mathrm{v2}}^2 + {\mathrm{v3}}^2\right)}^{\frac{3}{4}}} - \frac{\mathrm{dv3}\, {\mathrm{v3}}^2\, \sin\!\left(\sqrt{{\mathrm{v1}}^2 + {\mathrm{v2}}^2 + {\mathrm{v3}}^2}\right)}{{\left({\mathrm{v1}}^2 + {\mathrm{v2}}^2 + {\mathrm{v3}}^2\right)}^{\frac{5}{4}}} + \frac{2\, \mathrm{dv2}\, \mathrm{v1}\, {\sin\!\left(\frac{\sqrt{{\mathrm{v1}}^2 + {\mathrm{v2}}^2 + {\mathrm{v3}}^2}}{2}\right)}^2}{{\mathrm{v1}}^2 + {\mathrm{v2}}^2 + {\mathrm{v3}}^2} - \frac{2\, \mathrm{dv1}\, \mathrm{v2}\, {\sin\!\left(\frac{\sqrt{{\mathrm{v1}}^2 + {\mathrm{v2}}^2 + {\mathrm{v3}}^2}}{2}\right)}^2}{{\mathrm{v1}}^2 + {\mathrm{v2}}^2 + {\mathrm{v3}}^2} + \frac{\mathrm{dv1}\, \mathrm{v1}\, \mathrm{v3}}{{\left({\mathrm{v1}}^2 + {\mathrm{v2}}^2 + {\mathrm{v3}}^2\right)}^{\frac{3}{4}}} + \frac{\mathrm{dv2}\, \mathrm{v2}\, \mathrm{v3}}{{\left({\mathrm{v1}}^2 + {\mathrm{v2}}^2 + {\mathrm{v3}}^2\right)}^{\frac{3}{4}}} - \frac{\mathrm{dv1}\, \mathrm{v1}\, \mathrm{v3}\, \sin\!\left(\sqrt{{\mathrm{v1}}^2 + {\mathrm{v2}}^2 + {\mathrm{v3}}^2}\right)}{{\left({\mathrm{v1}}^2 + {\mathrm{v2}}^2 + {\mathrm{v3}}^2\right)}^{\frac{5}{4}}} - \frac{\mathrm{dv2}\, \mathrm{v2}\, \mathrm{v3}\, \sin\!\left(\sqrt{{\mathrm{v1}}^2 + {\mathrm{v2}}^2 + {\mathrm{v3}}^2}\right)}{{\left({\mathrm{v1}}^2 + {\mathrm{v2}}^2 + {\mathrm{v3}}^2\right)}^{\frac{5}{4}}} \end{array}\right)
\end{equation}

If one merely wishes to type in ordinary text, without
complicated mathematical formulae or special effects such
as font changes, then one merely has to type it in as it
is, leaving a completely blank line between successive
paragraphs.

You do not have to worry about paragraph indentation:
all paragraphs will be indented with the exception of
the first paragraph of a new section.

One must take care to distinguish between the `left quote'
and the `right quote' on the computer terminal.  Also, one
should use two `single quote' characters in succession if
one requires ``double quotes''.  One should never use the
(undirected) `double quote' character on the computer
terminal, since the computer is unable to tell whether it
is a `left quote' or a `right quote'.  One also has to
take care with dashes: a single dash is used for
hyphenation, whereas three dashes in succession are required
to produce a dash of the sort used for punctuation---such as
the one used in this sentence.

\section{Section Headings}

We explain in this section how to obtain headings
for the various sections and subsections of our
document.

\subsection{Headings in the `article' Document Style}

In the `article' style, the document may be divided up
into sections, subsections and subsubsections, and each
can be given a title, printed in a boldface font,
simply by issuing the appropriate command.

\end{document}
