\documentclass[a4paper,10pt]{article}
%\usepackage[landscape]{geometry}
\usepackage{color}
\usepackage{amsmath}
\usepackage{graphicx}
\usepackage[margin=1.5cm]{geometry}
\usepackage{pdflscape}
\usepackage{hyperref}
\usepackage{multirow}
\definecolor{orange}{rgb}{1,0.5,0}
\definecolor{opaqueblue}{rgb}{0.5,0.5,1.0}
\renewcommand{\baselinestretch}{1.2}

% User-defined macro for printing zero matrix in opaque color.
% Should be used in math blocks.
% Usage) \zm{2,3}
%        \zm{}
\newcommand{\zm}[1]{\ensuremath{ {\color{opaqueblue} 0_{#1} } }}

\begin{document}

%%%%%%%%% TITLE


\section{Appendix}

\subsection{Naming convention}
\begin{table}[h!b!p!]
\caption{Naming convention}
\centering
\begin{tabular}{ c c l }
\hline
Symbol           & Dimension                                & Name \\
\hline
$\bar{m}$        &                                          & body mass \\
$\mathbf{H}$     & $3 \times 3$                             & moments of inertia \\
$a$              & $3 \times 1$                             & linear acceleration \\
$\alpha$         & $3 \times 1$                             & angular acceleration \\
$\omega$         & $3 \times 1$                             & angular velocity \\
$f$              & $3 \times 1$                             & external force\\
$\tau$           & $3 \times 1$                             & external torque force\\
$n_d$            &                                          & degree-of-freedom of a rigid body\\
$n$              &                                          & the number of rigid bodies\\
$m$              &                                          & the number of muscle fibers\\
$\chi$           & $n_d n     \times 1$                     & generalized position\\
$\dot\chi$       & $n_d n     \times 1$                     & generalized velocity\\
$\ddot\chi$      & $n_d n     \times 1$                     & generalized acceleration\\
$\mathbf{M}$     & $n_d n     \times n_d$                   & generalized mass matrix\\
$n_Y$            &                                          & dimension of the state vector $Y$\\
$Y$              & $2n_d n + m\times 1$                     & generalized state vector\\
$h$              &                                          & simulation time step\\
$C$              & $2n_d n    \times 1$                     & Coriolis and centrifugal force\\
$k_{se}$         &                                          & serial spring constant\\
$k_{pe}$         &                                          & parallel spring constant\\
$x$              &                                          & muscle fiber length\\
$x_r$            &                                          & muscle fiber rest length\\
$b$              &                                          & muscle fiber viscosity\\
$T$              &                                          & muscle fiber tension\\
$A$              &                                          & muscle fiber actutation force\\
$\mathbf{W}_Y$   &     $n_Y \times n_Y$ (diagonal)   & weight for tracking\\
$\mathbf{W}_u$   &     $ m  \times  m $ (diagonal)   & weight for actuation force\\
\hline
\end{tabular}
\end{table}

\subsection{Linearization of constraints}

\subsubsection{Equations of motion}
The discretized version of \eqref{MoEq2} is the following:
\begin{equation}\label{Moeq_discretized}
\mathbf{M}(\chi^{(l)})\ddot\chi^{(l)} + C(\chi^{(l)}, \dot\chi^{(l)})
     = f_g(\chi^{(l)}) + f_c(\chi^{(l)}, \lambda^{(l)}) + f_m(\chi^{(l)}, T^{(l)})
\end{equation}
By substituting the definitions \eqref{vel-dis} and \eqref{acc-dis} into \eqref{Moeq_discretized}
we can rearrange this constraint equation into the linear function of
the optimization variable $\chi^{(l+1)}$.
\begin{equation}\label{Moeq_discretized2}
\mathbf{M}(\chi^{(l)}) \frac{\chi^{(l+1)}-2\chi^{(l)}+\chi^{(l-1)}}{h^2} + C(\chi^{(l)}, \frac{\chi^{(l)} - \chi^{(l-1)}}{h})
= f_g(\chi^{(l)}) + f_c(\chi^{(l)}, \lambda^{(l)}) + f_m(\chi^{(l)}, T^{(l)})
\end{equation}
$\mathbf{M}$, $C$ and $f_g$ can be completely determined at current time step $l$.
\begin{align*}\label{Moeq_nextstate}
\mathbf{M} \frac{\chi^{(l+1)}-2\chi^{(l)}+\chi^{(l-1)}}{h^2}  - f_c - f_m &= f_g - C \\
\chi^{(l+1)}-2\chi^{(l)}+\chi^{(l-1)}  - \mathbf{M}^{-1}h^2(f_c + f_m) &= \mathbf{M}^{-1}h^2(f_g - C)\\
\chi^{(l+1)}  - \mathbf{M}^{-1}h^2(f_c + f_m) &= 2\chi^{(l)}-\chi^{(l-1)} + \mathbf{M}^{-1}h^2(f_g - C)
\end{align*}
We need to linearize $f_c$ and $f_m$ subsequently.
Before we proceed, let us define some vector quantities used frequently
afterwards.
\begin{equation}
\chi =
\left[ \begin{array}{c}
\chi_{1}\\
\vdots\\
\chi_{n}
\end{array}  \right]
,\quad
\chi_i = \left [
\begin{array}{c}
p_i \\
v_i
\end{array}
\right ]
,\quad
\lambda =
\left[ \begin{array}{c}
c_{t,1}\\
c_{n,1}\\
\vdots\\
c_{t,|\mathcal{P}|}\\
c_{n,|\mathcal{P}|}\\
\end{array}  \right]
,\quad
T =
\left[ \begin{array}{c}
T_{1}\\
\vdots\\
T_{m}
\end{array}  \right]
,\quad
u =
\left[ \begin{array}{c}
u_1\\
\vdots\\
u_{m}
\end{array}  \right]
\end{equation}
Here $\mathcal{P}$ is a set of indices of contact points.
\subsubsection{Gravitational forces}
The gravitational force $f_g$ is a function of $\chi$ and other
physical constants such as mass, a center of mass and the gravitational
acceleration.
\begin{equation}
f_g(\chi) = \left [
\begin{array}{c}
f_{g,1}(\chi_1 )\\
\vdots\\
f_{g,n}(\chi_n )
\end{array}
\right ]
,\quad\quad
f_{g,i}(\chi_i) = \left [
\begin{array}{c}
\left( \frac{\partial\mathbf{W}_i(\chi_{i})}{\partial\chi_{i,1}}c_i \right) \cdot m_i g\\
\vdots\\
\left( \frac{\partial\mathbf{W}_i(\chi_{i})}{\partial\chi_{i,n_d}}c_i \right) \cdot m_i g
\end{array}
\right ]
\end{equation}

\subsubsection{Contact forces}
$f_c$ can be represented as $\mathbf{Q}\mathbf{S}_\mathcal{P}\lambda$ by using the following derivation.
\begin{equation}\label{f_c_lin}
f_c(\chi,\lambda) = \left [
\begin{array}{c}
f_{c,1}(\chi_1, \lambda )\\
\vdots\\
f_{c,n}(\chi_n, \lambda )
\end{array}
\right ]
=
\left [
\begin{array}{ccc}
\mathbf{Q}_1    &            &              \\
                &  \ddots    &              \\
                &            & \mathbf{Q}_n
\end{array}
\right ]
\left [
\begin{array}{c}
\mathbf{S}_{\mathcal{P}(1)}\\
\vdots\\
\mathbf{S}_{\mathcal{P}(n)}
\end{array}
\right ]
\lambda
=
\mathbf{Q} \mathbf{S}_\mathcal{P} \lambda
\end{equation}

\begin{align}
f_{c,i}(\chi_i,\lambda)
      & = \sum_{j \in \mathcal{P}(i)}\left [
\begin{array}{c}
\left( \frac{\partial\mathbf{W}_i(\chi_{i})}{\partial\chi_{i,1}}p_j \right) \cdot (c_{t,j} + n_j c_{n,j})\\
\vdots\\
\left( \frac{\partial\mathbf{W}_i(\chi_{i})}{\partial\chi_{i,n_d}}p_j \right) \cdot (c_{t,j} + n_j c_{n,j})
\end{array}
\right ]\notag\\
      & = \sum_{j \in \mathcal{P}(i)}\left [
\begin{array}{c}
\left( \frac{\partial\mathbf{W}_i(\chi_{i})}{\partial\chi_{i,1}}p_j \right)^T \\
\vdots\\
\left( \frac{\partial\mathbf{W}_i(\chi_{i})}{\partial\chi_{i,n_d}}p_j \right)^T
\end{array}
\right ]
\left [
\begin{array}{cc}
\mathbf{1}_4 \quad n_j
\end{array}
\right ]
\left [
\begin{array}{c}
c_{t,j}\\
c_{n,j}
\end{array}
\right ]\notag\\
       & = \sum_{j \in \mathcal{P}(i)} \mathbf{Q}_{i,j}
\left [
\begin{array}{c}
c_{t,j}\\
c_{n,j}
\end{array}
\right ]\notag\\
       & =
\left [
\begin{array}{ccc}
\mathbf{Q}_{i,\mathcal{P}(i)_1} &  \cdots  &  \mathbf{Q}_{i,\mathcal{P}(i)_{|\mathcal{P}(i)|}}
\end{array}
\right ]
\left [
\begin{array}{c}
c_{t,\mathcal{P}(i)_1}\\
c_{n,\mathcal{P}(i)_1}\\
\vdots \\
c_{t,\mathcal{P}(i)_{|\mathcal{P}(i)|}}\\
c_{n,\mathcal{P}(i)_{|\mathcal{P}(i)|}}
\end{array}
\right ]\notag \\
       & =
\mathbf{Q}_i \mathbf{S}_{\mathcal{P}(i)} \lambda
\end{align}

\begin{equation}
\mathbf{Q}_{i,j} =
\left [
\begin{array}{ccc}
\left( \frac{\partial\mathbf{W}_i(\chi_{i})}{\partial\chi_{i,1}}p_j \right)^T &  &  \left( \frac{\partial\mathbf{W}_i(\chi_{i})}{\partial\chi_{i,1}}p_j \right) \cdot n_j \\
& \vdots & \\
\left( \frac{\partial\mathbf{W}_i(\chi_{i})}{\partial\chi_{i,n_d}}p_j \right)^T &  &  \left( \frac{\partial\mathbf{W}_i(\chi_{i})}{\partial\chi_{i,n_d}}p_j \right) \cdot n_j
\end{array}
\right ] \quad\quad (6 \times 5)
\end{equation}
The contact normal force coefficient $c_{n,j}$ should be nonnegative since the force
should not pull the contact point along the normal direction.
Also the friction force vector $c_{t,j}$ need to be perpendicular to
the contact normal direction. Lastly, we need to ensure that the sum of
the normal and the friction force vector should reside in the Coulomb friction cone.
To satisfy these conditions we need the following constraints:
\begin{equation}\label{Contact_force_constraints}
c_{n,j}           \geq  0,               \quad\quad
n_j \cdot c_{t,j}   =   0,               \quad\quad
\| c_{t,j} \|_2   \leq  \mu _j c_{n,j}   \quad\quad  \forall j \in \mathcal{P}
\end{equation}



\subsubsection{Muscle forces}
$f_m$ can be represented as $\mathbf{R}\mathbf{S}_\mathcal{M} T$ by using the following derivation.
\begin{equation}\label{f_m_lin}
f_m(\chi, T) = \left [
\begin{array}{c}
f_{m,1}(\chi, T )\\
\vdots\\
f_{m,n}(\chi, T )
\end{array}
\right ]
=
\left [
\begin{array}{ccc}
\mathbf{R}_1                      &            &                               \\
                                  &  \ddots    &                               \\
                                  &            & \mathbf{R}_n
\end{array}
\right ]
\left [
\begin{array}{c}
\mathbf{S}_{\mathcal{M}(1)}  \\
\vdots  \\
\mathbf{S}_{\mathcal{M}(n)}
\end{array}
\right ]
T
=
\mathbf{R} \mathbf{S}_\mathcal{M} T
\end{equation}

\begin{align}
f_{m,i}(\chi ,T)
      & = \sum_{j \in \mathcal{M}(i)}\left [
\begin{array}{c}
\left( \frac{\partial\mathbf{W}_i(\chi_{i})}{\partial\chi_{i,1}} a_{i,j} \right) \cdot T_j d(i,j)\\
\vdots\\
\left( \frac{\partial\mathbf{W}_i(\chi_{i})}{\partial\chi_{i,n_d}} a_{i,j} \right) \cdot T_j d(i,j)
\end{array}
\right ]\notag\\
      & = \sum_{j \in \mathcal{M}(i)}\left [
\begin{array}{c}
\left( \frac{\partial\mathbf{W}_i(\chi_{i})}{\partial\chi_{i,1}} a_{i,j} \right) \cdot d(i,j) \\
\vdots\\
\left( \frac{\partial\mathbf{W}_i(\chi_{i})}{\partial\chi_{i,n_d}} a_{i,j} \right) \cdot d(i,j)
\end{array}
\right ]
T_j\notag\\
       & = \sum_{j \in \mathcal{M}(i)} \mathbf{R}_{i,j}
T_j\notag\\
       & =
\left [
\begin{array}{ccc}
\mathbf{R}_{i,\mathcal{M}(i)_1} &  \cdots  &  \mathbf{R}_{i,\mathcal{M}(i)_{|\mathcal{M}(i)|}}
\end{array}
\right ]
\left [
\begin{array}{c}
T_{\mathcal{M}(i)_1}\\
\vdots \\
T_{\mathcal{M}(i)_{|\mathcal{M}(i)|}}
\end{array}
\right ]\\
       & =
\mathbf{R}_i \mathbf{S}_{\mathcal{M}(i)} T
\end{align}

\subsubsection{Muscle fiber dynamics}
The discretized version of \eqref{Tension} is the following:
\begin{equation}
\frac{T^{(l)}-T^{(l-1)}}{h}
           = \frac{k_{se}}{b} \left( k_{pe}(x^{(l)}-x_{r})+b\frac{x^{(l)}-x^{(l-1)}}{h}-\left(1+\frac{k_{pe}}{k_{se}}\right) T^{(l)} + u^{(l)}   \right)
\end{equation}

\begin{equation}
k_{se} b (T^{(l)}-T^{(l-1)})
            = k_{se}^2 h k_{pe}(x^{(l)}-x_{r})
               + k_{se}^2 b (x^{(l)}-x^{(l-1)})
               -( k_{se}^2 h +k_{se} h k_{pe} ) T^{(l)}
               + k_{se}^2 h u^{(l)}
\end{equation}
\begin{equation}
k_{se} b  (T^{(l)}-T^{(l-1)}) + ( k_{se}^2 h +k_{se} h k_{pe} ) T^{(l)}
- k_{se}^2 h u^{(l)}
            = k_{se}^2 h k_{pe}(x^{(l)}-x_{r})
               + k_{se}^2 b (x^{(l)}-x^{(l-1)})
\end{equation}
\begin{equation}
(k_{se} b  + k_{se}^2 h +k_{se} h k_{pe} ) T^{(l)}
- k_{se}^2 h u^{(l)} =
k_{se} b T^{(l-1)}
            + k_{se}^2 h k_{pe}(x^{(l)}-x_{r})
               + k_{se}^2 b (x^{(l)}-x^{(l-1)})
\end{equation}
\begin{equation}
(k_{se} b  + k_{se}^2 h +k_{se} h k_{pe} ) T^{(l)}
- k_{se}^2 h u^{(l)} + k_{se}^2 h k_{pe} x^{(l)}_{r} =
k_{se} b T^{(l-1)}
            + k_{se}^2 h k_{pe} x^{(l)}
               + k_{se}^2 b (x^{(l)}-x^{(l-1)})
\end{equation}

\begin{equation}
\mathbf{K}_i
\left [
\begin{array}{c}
T^{(l)}_i \\ u^{(l)}_i \\ x_{r,i}^{(l)}
\end{array}
\right ] = s_i
\end{equation}

\begin{align}
\mathbf{K}_i & =
\left [
\begin{array}{ccc}
k_{se, i} b_i  + k_{se,i}^2 h +k_{se,i} h k_{pe,i}  &     - k_{se,i}^2 h  &  k_{se,i}^2 h k_{pe,i}
\end{array}
\right ]\\
s_i & = k_{se,i} b_i T^{(l-1)}_i
            + k_{se,i}^2 h k_{pe,i} x^{(l)}_i
               + k_{se,i}^2 b_i (x^{(l)}_i-x^{(l-1)}_i)
\end{align}

\begin{align}
\left [
\begin{array}{ccc}
\mathbf{K}_1 &        &                  \\
             & \ddots &                  \\
             &        & \mathbf{K}_{n_m}
\end{array}
\right ]
\left [
\begin{array}{c}
T_1^{(l)} \\ u_1^{(l)} \\ x_{r,1}^{(l)} \\
\vdots\\
T_{n_m}^{(l)} \\ u_{n_m}^{(l)} \\ x_{r,n_m}^{(l)}
\end{array}
\right ]  & =
\left [
\begin{array}{c}
s_1 \\
\vdots \\
s_{n_m}
\end{array}
\right ]\notag\\
\mathbf{K}\mathbf{S}_{T,u,x} \left [
\begin{array}{c}
T^{(l)} \\ u^{(l)} \\ x_{r}^{(l)}
\end{array}
\right ] &= s\label{Muscle_fiber_constraint}
\end{align}
This equation couples $T$, $u$ and $x_r$ with current state linearly
since $x$ is the function of $\chi$.

\subsection{Constraints for nonpenetration}
For each contact $j$ on the body $i$, we need a nonpenetrating inequality constraint which
takes a form of $C_{np}^j (\chi^{(l+1)}_i) \geq 0$.
If we assume a flat ground of $z=0$, $C_{np}^j$ is the same as
the $z$-axis coordinate of the contact point.
\begin{align}
C_{np}^j & = \left [ \begin{array}{cccc} 0 & 0 & 1 & 0 \end{array} \right ] p^{W(l+1)} \notag\\
         & = \left [ \begin{array}{cccc} 0 & 0 & 1 & 0 \end{array} \right ] ( \mathbf{Z}(i,p) \chi^{(l+1)}_i + V(i,p) ) \geq 0
\end{align}
where $p$ and $p^W$ are a contact point defined in body and world coordinates, respectively.

\subsection{Constraints for joints}
Assume that a joint $j$ exists between the body $a$ and $b$. Also assume that
$p_a$ and $p_b$ are body coordinate positions where a ligament fiber is attached
to each body. We want to make a constraint in order to confine a value $\|p_a^W-p_b^W\|$ within
a certain limit $\epsilon_j$.
\begin{align}
C_{joint}^j  & = \| p_a^W - p_b^W \| \notag\\
             & = \| ( \mathbf{Z}(a,p_a) \chi^{(l+1)}_a + V(a,p_a) ) - ( \mathbf{Z}(b,p_b) \chi^{(l+1)}_b + V(b,p_b) ) \| \leq \epsilon_j
\end{align}

\subsection{A complete constraints formulation}
If we substitute \eqref{f_c_lin} and \eqref{f_m_lin} into \eqref{Moeq_discretized2}
then the whole equations of motion becomes the linear function of the
optimization variables.
\begin{equation}\label{Moeq_discretized3}
\mathbf{M}(\chi^{(l)}) \frac{\chi^{(l+1)}-2\chi^{(l)}+\chi^{(l-1)}}{h^2} + C(\chi^{(l)}, \frac{\chi^{(l)} - \chi^{(l-1)}}{h})
= f_g(\chi^{(l)}) + \mathbf{Q} \mathbf{S}_\mathcal{P} \lambda^{(l)} + \mathbf{R} \mathbf{S}_\mathcal{M} T^{(l)}\notag
\end{equation}
By gathering the optimization variables to the left-hand side of the equation and
denoting it by matrix-vector notation we get the following equation:
\begin{align}\label{Moeq_constraint}
\frac{\mathbf{M}(\chi^{(l)})}{h^2}\chi^{(l+1)} - \mathbf{Q} \mathbf{S}_\mathcal{P} \lambda^{(l)} - \mathbf{R} \mathbf{S}_\mathcal{M} T^{(l)}
& = z \notag\\
\left [
\begin{array}{ccc}
\mathbf{M}(\chi^{(l)}) / h^2 & - \mathbf{Q} \mathbf{S}_\mathcal{P}  &  - \mathbf{R} \mathbf{S}_\mathcal{M}
\end{array}
\right ]
\left [
\begin{array}{c}
\chi^{(l+1)} \\
\lambda^{(l)} \\
T^{(l)}
\end{array}
\right ]
& = z
\end{align}
where
\begin{equation}
z \equiv f_g(\chi^{(l)}) - \mathbf{M}(\chi^{(l)})\frac{2\chi^{(l)}-\chi^{(l-1)}}{h^2} - C(\chi^{(l)}, \frac{\chi^{(l)} - \chi^{(l-1)}}{h}).\notag
\end{equation}
We can combine both of \eqref{Moeq_constraint} and \eqref{Muscle_fiber_constraint}
which are the equality constraints. To do this, let $\tau$ be the vector of aggregate
optimization variables and $\mathbf{X}_{\star}$ be the selection matrix,
\emph{i.e.} $\chi^{(l+1)} = \mathbf{X}_{\chi} \tau$.
The final form of the equality constraints is below:
\begin{equation}
\left [ \begin{array}{c}
\left [
\begin{array}{ccc}
\mathbf{M}(\chi^{(l)}) / h^2 & - \mathbf{Q} \mathbf{S}_\mathcal{P}  &  - \mathbf{R} \mathbf{S}_\mathcal{M}
\end{array}
\right ]
\left [
\begin{array}{c}
\mathbf{X}_\chi \\
\mathbf{X}_\lambda \\
\mathbf{X}_T
\end{array}
\right ]\\
\mathbf{K}\mathbf{S}_{T,u,x} \left [
\begin{array}{c}
\mathbf{X}_T \\ \mathbf{X}_u \\ \mathbf{X}_{x_r}
\end{array}
\right ]
\end{array}
\right ]
\tau
=
\left [
\begin{array}{c}
z\\
s
\end{array}
\right ]
\quad\quad
\text{where}
\quad\quad
\tau \equiv
\left [
\begin{array}{c}
\chi^{(l+1)} \\
\lambda^{(l)} \\
T^{(l)} \\
u^{(l)} \\
x_r^{(l)}
\end{array}
\right ].
\end{equation}
Moreover, contact force constraints \eqref{Contact_force_constraints}
need to be considered.

\subsection{Cost function}

\subsection{Linearization of next step states}
A point $p$ attached to a certain rigid body can be transformed into
world coordinates by multiplying a homogeneous transformation matrix $\mathbf{W}$
of the rigid body.
We can represent the next time position of $p$ in world coodinates
as the linear function of the next time state $\chi^{(l+1)}$
by using the following formula:
\begin{align}
\mathbf{W}^{(l+1)}p
        & \approx \mathbf{W}^{(l)}p + h\dot{\mathbf{W}}^{(l+1)}p\notag\\
        &    =    \mathbf{W}^{(l)}p + h\frac{\partial\mathbf{W}^{(l+1)}}{\partial\chi^{(l+1)}}\dot\chi^{(l+1)} p\notag\\
        & \approx \mathbf{W}^{(l)}p + h\frac{\partial\mathbf{W}^{(l  )}}{\partial\chi^{(l  )}}\dot\chi^{(l+1)} p\notag\\
        &    =    \mathbf{W}^{(l)}p + h\sum^{n_d}_{j=1}{ \frac{\partial\mathbf{W}^{(l  )}}{\partial\chi^{(l  )}_j}\dot\chi^{(l+1)}_j} p\notag\\
        &    =    \mathbf{W}^{(l)}p + h\sum^{n_d}_{j=1}{ \frac{\partial\mathbf{W}^{(l  )}}{\partial\chi^{(l  )}_j}\frac{\chi^{(l+1)}_j-\chi^{(l)}_j}{h}} p\notag\\
        &    =    \mathbf{W}^{(l)}p +  \sum^{n_d}_{j=1}{ \frac{\partial\mathbf{W}^{(l  )}}{\partial\chi^{(l  )}_j} p (\chi^{(l+1)}_j-\chi^{(l)}_j})\notag\\
        &    =    \mathbf{W}^{(l)}p + \mathbf{Z}\chi^{(l+1)} - \mathbf{Z}\chi^{(l)}\notag\\
        &    =    \mathbf{Z}\chi^{(l+1)} + V
\end{align}
where
\begin{equation}
\mathbf{Z} = \left[ \begin{array}{ccc}
                    \frac{\partial\mathbf{W}^{(l  )}}{\partial\chi^{(l  )}_1} p & \cdots & \frac{\partial\mathbf{W}^{(l  )}}{\partial\chi^{(l  )}_{n_d}} p
              \end{array} \right], \quad\quad
V = \mathbf{W}^{(l)}p - \mathbf{Z}\chi^{(l)}\notag
\end{equation}

\pagebreak

\begin{landscape}
\subsection{A particle with flat ground}
\begin{align}
\text{given}      & \quad \chi^{(l-1)}, \chi^{(l)}, f_g, m, h, \mu\notag\\
\text{minimize}   & \quad \epsilon + 100\chi^{(l)}_z f_{c,z} + \epsilon_{fric} \notag\\
\text{subject to} & \quad\quad\quad\quad\quad\quad\quad\quad\quad\quad\quad\quad\quad\quad\quad
\quad\quad\quad\quad\quad\quad\quad\quad\quad\quad\quad\quad\quad\quad\quad
\quad\quad\quad\quad\quad\quad\quad\quad\quad\quad\quad\quad\quad\quad\quad \notag
\end{align}
\begin{equation}
\left[\begin{array}{ccccccc}
0 & m/h^2\mathbf{1}     & -\mathbf{1}  & \mathbf{0} & 0 & 0 & 0 \\
0 &    \mathbf{1}       & \mathbf{0}   & \mathbf{1} & 0 & 0 & 0 \\
0 & [ 0, 0, \sqrt{m}/h] & \mathbf{0}   & \mathbf{0} & -1 & 0 & 0 \\
0 & \mathbf{0}          & [ 0, 0, \mu] & \mathbf{0} & 0 & 0 & -1
\end{array}\right]
\left[\begin{array}{c}
\epsilon \\ \chi^{(l+1)} \\ f_c \\ \Delta\chi \\ \tilde{c}_{lcp} \\ \epsilon_{fric} \\ \mu f_{c,z}
\end{array}\right]
=
\left[\begin{array}{c}
m/h^2(2\chi^{(l)}-\chi^{(l-1)}) + f_g \\
\chi^{(l)} \\
-hq/2\sqrt{m} \\
0
\end{array}\right]
\end{equation}
\begin{equation}
q \equiv -m/h^2(2\chi^{(l)}_z-\chi^{(l-1)}_z)-f_{g,z}
\end{equation}
\begin{equation}
\chi^{(l+1)}_z \geq 0       , \quad\quad
f_{c,z} \geq 0              , \quad\quad
\tilde{c}_{lcp} \geq 0          , \quad\quad
\epsilon_{fric} \geq 0          , \quad\quad
\mu f_{c,z} \geq 0
\end{equation}
\begin{equation}
\epsilon            \geq \| \tilde{c}_{lcp} \| , \quad\quad
\mu f_{c,z}  \geq \| f_{c,t}         \| , \quad\quad
\epsilon_{fric}    \geq \| \Delta\chi      \|
\end{equation}
\end{landscape}

\begin{landscape}
\subsection{Multiple particles with flat ground}
\begin{align}
\text{given}      & \quad \chi^{(l-1)}_1, \chi^{(l)}_1, f_{g,1}, m_1,\dots
                          \chi^{(l-1)}_n, \chi^{(l)}_n, f_{g,n}, m_n,
                          h, \mu \notag\\
\text{minimize}   & \quad \sum_{i=1}^n \epsilon_i + 100\chi^{(l)}_{i,z} f_{c,i,z} + \epsilon_{fric,i} \notag\\
\text{subject to} & \quad\quad\quad\quad\quad\quad\quad\quad\quad\quad\quad\quad\quad\quad\quad
\quad\quad\quad\quad\quad\quad\quad\quad\quad\quad\quad\quad\quad\quad\quad
\quad\quad\quad\quad\quad\quad\quad\quad\quad\quad\quad\quad\quad\quad\quad \notag
\end{align}
\begin{equation}
\left[\begin{array}{ccc}
\mathbf{A}_1 &         &  \\
             & \ddots  &  \\
             &         & \mathbf{A}_n
\end{array}\right]
\left[\begin{array}{c}
\tau_1  \\
 \vdots \\
\tau_n
\end{array}\right]
=
\left[\begin{array}{c}
\eta_1  \\
 \vdots \\
\eta_n
\end{array}\right]
\end{equation}
\begin{equation}
\mathbf{A}_i \equiv
\left[\begin{array}{ccccccc}
0 & m_i/h^2\mathbf{1}     & -\mathbf{1}   & \mathbf{0} & 0 & 0 & 0 \\
0 &    \mathbf{1}         & \mathbf{0}    & \mathbf{1} & 0 & 0 & 0 \\
0 & [ 0, 0, \sqrt{m_i}/h] & \mathbf{0}    & \mathbf{0} & -1 & 0 & 0 \\
0 & \mathbf{0}            & [ 0, 0, \mu ] & \mathbf{0} & 0 & 0 & -1
\end{array}\right]
\end{equation}
\begin{equation}
\tau_i \equiv
\left[\begin{array}{c}
\epsilon_i \\ \chi^{(l+1)}_i \\ f_{c,i} \\ \Delta\chi_i \\ \tilde{c}_{lcp,i} \\ \epsilon_{fric,i} \\ \mu f_{c,i,z}
\end{array}\right], \quad
\eta_i \equiv
\left[\begin{array}{c}
m_i/h^2(2\chi^{(l)}_i-\chi^{(l-1)}_i) + f_{g,i} \\
\chi^{(l)}_i \\
-hq_i/2\sqrt{m_i} \\
0
\end{array}\right]
\end{equation}
\begin{equation}
q_i \equiv -m_i/h^2(2\chi^{(l)}_{i,z}-\chi^{(l-1)}_{i,z})-f_{g,i,z}
\end{equation}
\begin{equation}
\chi^{(l+1)}_{i,z} \geq 0       , \quad\quad
f_{c,i,z} \geq 0              , \quad\quad
\tilde{c}_{lcp,i} \geq 0          , \quad\quad
\epsilon_{fric,i} \geq 0          , \quad\quad
\mu f_{c,i,z} \geq 0
\end{equation}
\begin{equation}
\epsilon_i            \geq \| \tilde{c}_{lcp,i} \| , \quad\quad
\mu f_{c,i,z}  \geq \| f_{c,i,t}         \| , \quad\quad
\epsilon_{fric,i}    \geq \| \Delta\chi_i      \|
\end{equation}
\end{landscape}

\begin{landscape}
\subsection{Two particles connected by a fiber with flat ground}
\begin{align}
\text{given}      & \quad \chi^{(l-1)}_1, \chi^{(l)}_1, f_{g,1}, m_1,
                          \chi^{(l-1)}_2, \chi^{(l)}_2, f_{g,2}, m_2,
                          h, \mu \notag\\
\text{minimize}   & \quad \epsilon_T + \epsilon_u + \sum_{i=1}^2 \epsilon_i + 100\chi^{(l)}_{i,z} f_{c,i,z} + \epsilon_{fric,i} \notag\\
\text{subject to} & \quad\quad\quad\quad\quad\quad\quad\quad\quad\quad\quad\quad\quad\quad\quad
\quad\quad\quad\quad\quad\quad\quad\quad\quad\quad\quad\quad\quad\quad\quad
\quad\quad\quad\quad\quad\quad\quad\quad\quad\quad\quad\quad\quad\quad\quad \notag
\end{align}
\begin{equation}
\left[\begin{array}{ccccccc}
\mathbf{A}_1                    &                                  &                                  &        &        &  0 & 0 \\
                                & \mathbf{A}_2                     &                                  &        &        &  0 & 0 \\
\space[ \mathbf{0},\mathbf{1} ] &                                  &  -(\chi^{(l)}_2 - \chi^{(l)}_1)  &        &        &  0 & 0 \\
                                & \space[ \mathbf{0},\mathbf{1} ]  &  -(\chi^{(l)}_1 - \chi^{(l)}_2)  &        &        &  0 & 0 \\
                                &                                  &                   K_{11}         & K_{12} & K_{13} &  0 & 0
\end{array}\right]
\left[\begin{array}{c}
\tau_1  \\
\tau_2  \\
T \\
u \\
x_r \\
\epsilon_T \\
\epsilon_u
\end{array}\right]
=
\left[\begin{array}{c}
\eta_1  \\
\eta_2  \\
\vec{0} \\
\vec{0} \\
s
\end{array}\right]
\end{equation}
\begin{equation}
x_{rl} \leq x_r \leq x_{ru}, \quad \epsilon_T \geq \| T \|, \quad \epsilon_u \geq \| u \|
\end{equation}
\begin{align}
\text{where } \forall i \in \mathcal{B} & \quad\quad\quad\quad\quad\quad\quad\quad\quad\quad\quad\quad\quad\quad\quad
\quad\quad\quad\quad\quad\quad\quad\quad\quad\quad\quad\quad\quad\quad\quad
\quad\quad\quad\quad\quad\quad\quad\quad\quad\quad\quad\quad\quad\quad\quad \notag
\end{align}
\begin{equation}
\mathbf{A}_i \equiv
\left[\begin{array}{cccccccc}
0 & m_i/h^2\mathbf{1}     & -\mathbf{1}   & \mathbf{0} & 0   & 0  & 0  & -\mathbf{1}           \\
0 &    \mathbf{1}         & \mathbf{0}    & \mathbf{1} & 0   & 0  & 0  & \mathbf{0}            \\
0 & [ 0, 0, \sqrt{m_i}/h] & \mathbf{0}    & \mathbf{0} & -1  & 0  & 0  & [0,0,-h/2\sqrt{m}_i]  \\
0 & \mathbf{0}            & [ 0, 0, \mu ] & \mathbf{0} & 0   & 0  & -1 & \mathbf{0}
\end{array}\right]
\end{equation}
\begin{equation}
\tau_i \equiv
\left[\begin{array}{c}
\epsilon_i \\ \chi^{(l+1)}_i \\ f_{c,i} \\ \Delta\chi_i \\ \tilde{c}_{lcp,i} \\ \epsilon_{fric,i} \\ \mu f_{c,i,z} \\ f_{T,i}
\end{array}\right], \quad
\eta_i \equiv
\left[\begin{array}{c}
m_i/h^2(2\chi^{(l)}_i-\chi^{(l-1)}_i) + f_{g,i} \\
\chi^{(l)}_i \\
-hq_i/2\sqrt{m_i} \\
0
\end{array}\right]
\end{equation}
\begin{equation}
q_i \equiv -m_i/h^2(2\chi^{(l)}_{i,z}-\chi^{(l-1)}_{i,z})-f_{g,i,z}
\end{equation}
\begin{equation}
\chi^{(l+1)}_{i,z} \geq 0       , \quad\quad
f_{c,i,z} \geq 0              , \quad\quad
\tilde{c}_{lcp,i} \geq 0          , \quad\quad
\epsilon_{fric,i} \geq 0          , \quad\quad
\mu f_{c,i,z} \geq 0
\end{equation}
\begin{equation}
\epsilon_i            \geq \| \tilde{c}_{lcp,i} \| , \quad\quad
\mu f_{c,i,z}  \geq \| f_{c,i,t}         \| , \quad\quad
\epsilon_{fric,i}    \geq \| \Delta\chi_i      \|
\end{equation}
\end{landscape}

\pagebreak

\begin{landscape}
\subsection{A rigid body with flat ground}
\begin{align}
\text{given}      & \quad \chi^{(l-1)}, \chi^{(l)}, f_g, \mathbf{M}, C, h, \mu, p_{cfix}^{(l)}, \mathbf{Z}, V, \mathbf{Q}\notag\\
\text{minimize}   & \quad \sum_{j\in\mathcal{P}} (\tilde{p}_{c}^{(l+1)})_{jz} + (\epsilon_{c})_j \notag\\
\text{subject to} & \quad\quad\quad\quad\quad\quad\quad\quad\quad\quad\quad\quad\quad\quad\quad
\quad\quad\quad\quad\quad\quad\quad\quad\quad\quad\quad\quad\quad\quad\quad
\quad\quad\quad\quad\quad\quad\quad\quad\quad\quad\quad\quad\quad\quad\quad \notag
\end{align}
\begin{equation}\notag
\bordermatrix{
                                    &    \scriptstyle{(0) n_d}                      &              \scriptstyle{(1) n_d |\mathcal{P}|}          &        \scriptstyle{(2) 5 |\mathcal{P}|}       &  \scriptstyle{(3) 4 |\mathcal{P}|}   & \scriptstyle{(4) 4|\mathcal{P}|}    &   \scriptstyle{(5) |\mathcal{P}| }  & \scriptstyle{(6) |\mathcal{P}| }  & \scriptstyle{(7) n_d}  &   \scriptstyle{(8) 1 } \cr
\scriptstyle{(0) n_d}               &  \mathbf{M}/h^2                               & (-\mathbf{1})_{j \in \mathcal{P}}^\text{ce}               &                                                &                                      &                                     &                                     &                                   &                        &                        \cr
\scriptstyle{(1) n_d |\mathcal{P}|} &                                               &  -\mathbf{1}                                              &  (\mathbf{Q}_j)_{j \in \mathcal{P}}^\text{de}  &                                      &                                     &                                     &                                   &                        &                        \cr
\scriptstyle{(2) 4 |\mathcal{P}|}   &                                               &                                                           &                                                &  \mathbf{1}                          &  -\mathbf{1}                        &                                     &                                   &                        &                        \cr
\scriptstyle{(3)   |\mathcal{P}|}   &                                               & (\mathbf{c}_1)_{j \in \mathcal{P}}^\text{de}              &                                                &                                      &                                     &                                     & -\mathbf{1}                       &                        &                        \cr
\scriptstyle{(4) 4 |\mathcal{P}|}   &  (\mathbf{Z}_j)_{j \in \mathcal{P}}^\text{re} &                                                           &                                                &  -\mathbf{1}                         &                                     &                                     &                                   &                        &                        \cr
\scriptstyle{(5) n_d}               &  \mathbf{1}                                   &                                                           &                                                &                                      &                                     &                                     &                                   & -\mathbf{1}            &
}
\begin{pmatrix}
\chi^{(l+1)} \\
f_{c} \\
c_{c} \\
\tilde{p}^{(l+1)}_{c} \\
\Delta \tilde{p}_{c} \\
\epsilon_{c} \\ \mu f_{c,z} \\
\Delta\chi \\
\epsilon_{\Delta}
\end{pmatrix}
=
\begin{pmatrix}
c_2 \\
0 \\
p_{cfix}^{(l)} \\
0 \\
(-V_{j})_{j \in \mathcal{P}}^\text{re} \\
\chi^{(l)}
\end{pmatrix}
\end{equation}
\begin{equation}\notag
(\tilde{p}^{(l+1)}_{c})_z \succeq 0       , \quad\quad
(f_c)_z \succeq 0              , \quad\quad
\epsilon_{c} \succeq 0          , \quad\quad
\mu f_{c,z} \succeq 0
\end{equation}
\begin{equation}\notag
(\tilde{p}_c^{(l+1)})_{j4} = 1 , \quad\quad
%(\Delta \tilde{p}_c)_{j4} = 1 , \quad\quad
(c_c)_{j4} = 0  \quad\quad \forall j \in \mathcal{P}
\end{equation}
\begin{equation}\notag
(\mu f_{c,z})_j  \geq \| (f_{c,t})_j         \| , \quad\quad
(\epsilon_{c})_j    \geq \| (\Delta \tilde{p}_{c})_j      \| \quad\quad \forall j \in \mathcal{P}
\end{equation}
\begin{align*}
\mathbf{c}_1      & \equiv [ 0, 0, \mu, 0, 0, 0 ]\\
c_2               & \equiv \mathbf{M}/h^2(2\chi^{(l)}-\chi^{(l-1)}) - C + f_g
\end{align*}
\end{landscape}

\pagebreak

\newcommand{\eyemuscle}{\ensuremath{
\begin{bmatrix}
\mathbf{0} & -\mathbf{1}
\end{bmatrix}
}}

\begin{landscape}
\subsection{Multiple rigid bodies connected with multiple fibers}
\[Version 200\]
The matrix-vector notation for equility constraints:
\begin{equation}\notag
\bordermatrix{
                                        & \scriptstyle{0      }        &   \scriptstyle{1 : |\mathcal{M}|}           &  \scriptstyle{2 : |\mathcal{M}|}           & \scriptstyle{3 : |\mathcal{M}|}            &  \scriptstyle{4 : 1}    & \scriptstyle{5 : 1}    &  \scriptstyle{6:4| \mathcal{J} | }  &  \scriptstyle{7 : | \mathcal{J} |}     &  \scriptstyle{8 : 3} &  \scriptstyle{9:3} & \scriptstyle{10:1}  &  \scriptstyle{11:3}  & \scriptstyle{12:1} & \scriptstyle{13:1}& \scriptstyle{14:1}& \scriptstyle{15:|\mathcal{M}|}& \scriptstyle{16:|\mathcal{M}|}& \scriptstyle{17:3}& \scriptstyle{18:1} \cr
\scriptstyle{0      }                   &\mathbf{A}                    &                                             &                                            &                                            &                         &                        &                                     &                                        &                      &                    &                     &                      &                    &                   &                   &                               &                               &                   &                    \cr
\scriptstyle{1 : 2n_d|\mathcal{M}|}     &\mathbf{E}                    &   \mathbf{R}                                &                                            &                                            &                         &                        &                                     &                                        &                      &                    &                     &                      &                    &                   &                   &                               &                               &                   &                    \cr
\scriptstyle{2 : |\mathcal{M}|}         &                              &   \mathbf{K}_{11}                           &  \mathbf{K}_{12}                           & \mathbf{K}_{13}                            &                         &                        &                                     &                                        &                      &                    &                     &                      &                    &                   &                   &                               &                               &                   &                    \cr
\scriptstyle{3 : 4| \mathcal{J} | }     &\mathbf{D}                    &                                             &                                            &                                            &                         &                        &  -\mathbf{1}                        &                                        &                      &                    &                     &                      &                    &                   &                   &                               &                               &                   &                    \cr
\scriptstyle{4 : | \mathcal{J} | }      &                              &                                             &                                            &                                            &                         &                        &                                     &  \mathbf{0}                            &                      &                    &                     &                      &                    &                   &                   &                               &                               &                   &                    \cr
\scriptstyle{5 : 3 }                    &  \mathbf{A}_\text{com}       &                                             &                                            &                                            &                         &                        &                                     &                                        &      \mathbf{-1} M   &                    &                     &                      &                    &                   &                   &                               &                               &                   &                    \cr
\scriptstyle{6 : 3 }                    &                              &                                             &                                            &                                            &                         &                        &                                     &                                        &      \mathbf{1}      &  \mathbf{-1}       &                     &                      &                    &                   &                   &                               &                               &                   &                    \cr
\scriptstyle{7 : 3 }                    & \mathbf{G}                   &                                             &                                            &                                            &                         &                        &                                     &                                        &                      &                    &                     & \mathbf{-1}          &                    &                   &                   &                               &                               &                   &                    \cr
\scriptstyle{8 : 3 }                    & \mathbf{E}_2                 &                                             &                                            &                                            &                         &                        &                                     &                                        &                      &                    &                     &                      &                    &                   &                   &                               &                               &\mathbf{-1}        &                    \cr
}
\bordermatrix{
                 &                             \cr
\scriptstyle{ 0} & \tau                        \cr
\scriptstyle{ 1} & T                           \cr
\scriptstyle{ 2} & u                           \cr
\scriptstyle{ 3} & x_r                         \cr
\scriptstyle{ 4} & \epsilon_\text{ligten}      \cr
\scriptstyle{ 5} & \epsilon_\text{actten}      \cr
\scriptstyle{ 6} & d_A                         \cr
\scriptstyle{ 7} & \epsilon_d                  \cr
\scriptstyle{ 8} & p_\text{com}^{(l+1)}        \cr
\scriptstyle{ 9} & \Delta p_{\text{com,ref}}   \cr
\scriptstyle{10} & \epsilon_{\text{com}}       \cr
\scriptstyle{11} & \tau_{com}                  \cr
\scriptstyle{12} & \epsilon_{\tau,\text{com}}  \cr
\scriptstyle{13} & \epsilon_\text{ligact}      \cr
\scriptstyle{14} & \epsilon_\text{actact}      \cr
\scriptstyle{15} & \epsilon_T                  \cr
\scriptstyle{16} & \epsilon_u                  \cr
\scriptstyle{17} & f_\text{comfdev}            \cr
\scriptstyle{18} & \epsilon_\text{comfdev}     \cr
}
=
\bordermatrix{
                 &                             \cr
\scriptstyle{ 0} & \eta                        \cr
\scriptstyle{ 1} & \vec{0}                     \cr
\scriptstyle{ 2} & s                           \cr
\scriptstyle{ 3} & \vec{0}                     \cr
\scriptstyle{ 4} & \vec{0}                     \cr
\scriptstyle{ 5} & \vec{0}                     \cr
\scriptstyle{ 6} & p_\text{com,ref}^{(l+1)}    \cr
\scriptstyle{ 7} & \vec{0}                     \cr
\scriptstyle{ 8} & q                           \cr
}
\end{equation}
\begin{equation}\notag
\tau \equiv (\tau_i)^\text{re}_{i \in \mathcal{B}}, \quad
T          \equiv (T_i)^\text{re}_{i \in \mathcal{M}}, \quad
u          \equiv (u_i)^\text{re}_{i \in \mathcal{M}}, \quad
x_r        \equiv (x_{r,i})^\text{re}_{i \in \mathcal{M}}, \quad
d_A        \equiv (d_{A,i})^\text{re}_{i \in \mathcal{J}}, \quad
\epsilon_d \equiv (\epsilon_{d,i})^\text{re}_{i \in \mathcal{J}}
\end{equation}
\begin{equation}\notag
\mathbf{A} \equiv (\mathbf{A}_i)^\text{de}_{i \in \mathcal{B}}, \quad
\mathbf{R} \equiv (\mathbf{R}_i)^\text{re}_{i \in \mathcal{B}}, \quad
\mathbf{E} \equiv (\tilde{\mathbf{E}}_i)^\text{de}_{i \in \mathcal{B}}
\end{equation}
\begin{equation}\notag
\mathbf{K}_{11} \equiv (K_{11,j})^{\text{de}}_{j \in \mathcal{M}}, \quad
\mathbf{K}_{12} \equiv (K_{12,j})^{\text{de}}_{j \in \mathcal{M}}, \quad
\mathbf{K}_{13} \equiv (K_{13,j})^{\text{de}}_{j \in \mathcal{M}}
\end{equation}
\begin{equation}\notag
\eta             \equiv (\eta_i)^\text{re}_{i \in \mathcal{B}}, \quad
s                \equiv (s_j)^{\text{re}}_{j \in \mathcal{M}}, \quad
d_\text{disloc}  \equiv (d_{\text{disloc},i})^{\text{re}}_{i \in \mathcal{J}}
\end{equation}
\begin{equation}\notag
\mathbf{R}_i : n_d | \mathcal{M}(i) | \times |\mathcal{M}|, \quad
\tilde{\mathbf{E}}_i = [\mathbf{0},-\mathbf{1},\mathbf{0}_{n_d | \mathcal{M}(i) | \times 4|\mathcal{A}(i)|}] \quad : n_d | \mathcal{M}(i) | \times \text{colsize}(\mathbf{A}_i)
\end{equation}
\begin{equation}\notag
\mathbf{A}_\text{com} = ([ \mathbf{1}m_j, \mathbf{0} ])^{ce}_{j \in \mathcal{B}}
\end{equation}

\noindent Inequality constraints:
\begin{equation}\notag
x_{rl,j} \preceq x_{r,j} \preceq x_{ru,j} \quad \forall j\in \mathcal{M}
\end{equation}

\noindent Second-order cone constraints:
\begin{equation}\notag
\epsilon_\text{lig} \geq \| u_\text{lig} \|, \quad \epsilon_\text{act} \geq \| u_\text{act} \|, \quad
\epsilon_{d,i} \geq \| d_{A,i} \| \quad \forall i \in \mathcal{J}
\end{equation}

\[Version 200\]
\noindent The equility constraints $\mathbf{A}_i \tau_i = \eta_i \quad \forall i \in \mathcal{B}$ :

Here $\mathcal{P}$, $\mathcal{M}$ and $\mathcal{A}$ are specific for body $i$,
\emph{i.e.} $\mathcal{P}(i)$, $\mathcal{M}(i)$ and $\mathcal{A}(i)$, respectively.
\begin{equation}\notag
\mathbf{A}_i \tau_i
\equiv
\bordermatrix{
                                     &    \scriptstyle{0 : n_d}                      &              \scriptstyle{1 : n_d |\mathcal{P}|}          &        \scriptstyle{2 : 5 |\mathcal{P}|}               &  \scriptstyle{3 : 4 |\mathcal{P}|}                      & \scriptstyle{4 : 4|\mathcal{P}|}    &   \scriptstyle{5 : |\mathcal{P}| }  & \scriptstyle{6 : |\mathcal{P}| }  & \scriptstyle{7 : n_d}  &   \scriptstyle{8 : 1 } & \scriptstyle{9 : n_d |\mathcal{M}| }             & \scriptstyle{10: 4 |\mathcal{A}| }    &  \scriptstyle{11:1}        & \scriptstyle{12:n_d}   & \scriptstyle{13:1}   & \scriptstyle{14:|\mathcal{P}|}  & \scriptstyle{15:|\mathcal{P}|}   &  \scriptstyle{16:n_d}    \cr
\scriptstyle{ 0:n_d}                 &  \mathbf{M}/h^2                               & (-\mathbf{1})_{j \in \mathcal{P}}^\text{ce}               &                                                        &                                                         &                                     &                                     &                                   &                        &                        & (-\mathbf{1})_{j \in \mathcal{M}}^\text{ce}      &                                       &                            &                        &                      &                                 &                                  &      \mathbf{-1}         \cr
\scriptstyle{ 1:n_d |\mathcal{P}|}   &                                               &  -\mathbf{1}                                              &  (\mathbf{Q}_j)_{j \in \mathcal{P}}^\text{de}          &                                                         &                                     &                                     &                                   &                        &                        &                                                  &                                       &                            &                        &                      &                                 &                                  &                          \cr
\scriptstyle{ 2:4 |\mathcal{P}|}     &                                               &                                                           &                                                        &  \mathbf{1}                                             &  -\mathbf{1}                        &                                     &                                   &                        &                        &                                                  &                                       &                            &                        &                      &                                 &                                  &                          \cr
\scriptstyle{ 3:  |\mathcal{P}|}     &                                               &                                                           &(\mathbf{C}_\mu)_{j \in \mathcal{P}}^\text{de}          &                                                         &                                     &                                     & -\mathbf{1}                       &                        &                        &                                                  &                                       &                            &                        &                      &                                 &                                  &                          \cr
\scriptstyle{ 4: 4 |\mathcal{P}|}    &  (\mathbf{Z}_j)_{j \in \mathcal{P}}^\text{re} &                                                           &                                                        &  -\mathbf{1}                                            &                                     &                                     &                                   &                        &                        &                                                  &                                       &                            &                        &                      &                                 &                                  &                          \cr
\scriptstyle{ 5: n_d}                &  \mathbf{1}                                   &                                                           &                                                        &                                                         &                                     &                                     &                                   & -\mathbf{1}            &                        &                                                  &                                       &                            &                        &                      &                                 &                                  &                          \cr
\scriptstyle{ 6: |\mathcal{P}|}      &                                               &                                                           &  (\mathbf{C}_n)_{j \in \mathcal{P}}^\text{de}          &                                                         &                                     &                                     &                                   &                        &                        &                                                  &                                       &                            &                        &                      &                                 &                                  &                          \cr
\scriptstyle{ 7: 4 |\mathcal{A}|}    &  (\mathbf{Z}_j)_{j \in \mathcal{A}}^\text{re} &                                                           &                                                        &                                                         &                                     &                                     &                                   &                        &                        &                                                  &  -\mathbf{1}                          &                            &                        &                      &                                 &                                  &                          \cr
\scriptstyle{ 8: n_d}                &          \mathbf{1}                           &                                                           &                                                        &                                                         &                                     &                                     &                                   &                        &                        &                                                  &                                       &                            &       -\mathbf{1}      &                      &                                 &                                  &                          \cr
\scriptstyle{ 9: |\mathcal{P}|}      &                                               &                                                           &  (\mathbf{C}_\text{ns})_{j \in \mathcal{P}}^\text{de}  & (\mathbf{C}_\text{ns2})_{j \in \mathcal{P}}^\text{de}   &                                     &                                     &                                   &                        &                        &                                                  &                                       &                            &                        &                      &                                 &   -\mathbf{1}                    &                          \cr
\scriptstyle{10: |\mathcal{P}|}      &                                               &                                                           &  (\mathbf{C}_\text{tfx})_{j \in \mathcal{P}}^\text{de} &                                                         &                                     &                                     &                                   &                        &                        &                                                  &                                       &                            &                        &                      &                                 &                                  &                          \cr
\scriptstyle{11: |\mathcal{P}|}      &                                               &                                                           &  (\mathbf{C}_\text{tfy})_{j \in \mathcal{P}}^\text{de} &                                                         &                                     &                                     &                                   &                        &                        &                                                  &                                       &                            &                        &                      &                                 &                                  &                          \cr
\scriptstyle{12: 3|\mathcal{D}|}     &                                               &                                                           &                                                        &                                                         &                                     &                                     &                                   &                        &                        &                                                  &                                       &                            &                        &                      &                                 &                                  &                          \cr
}
\bordermatrix {
                 &                                     \cr
\scriptstyle{0}  &   \chi^{(l+1)}_i                    \cr
\scriptstyle{1}  &   f_{c,i}                           \cr
\scriptstyle{2}  &   c_{c,i}                           \cr
\scriptstyle{3}  &   \tilde{p}^{(l+1)}_{c,i}           \cr
\scriptstyle{4}  &   \Delta \tilde{p}_{c,i}            \cr
\scriptstyle{5}  &   \epsilon_{c,i}                    \cr
\scriptstyle{6}  &   \mu c_{n,i}                       \cr
\scriptstyle{7}  &   \Delta\chi_{i,\text{ref}}         \cr
\scriptstyle{8}  &   \epsilon_{\Delta,i,\text{ref}}    \cr
\scriptstyle{9}  &   f_{T,i}                           \cr
\scriptstyle{10} &   \tilde{p}^{(l+1)}_{A,i}           \cr
\scriptstyle{11} &   \epsilon_{\text{rot}}             \cr
\scriptstyle{12} &   \Delta\chi_{i,\text{prv}}         \cr
\scriptstyle{13} &   \epsilon_{\Delta,i,\text{prv}}    \cr
\scriptstyle{14} &   \epsilon_{p,z}                    \cr
\scriptstyle{15} &   \kappa_\text{compen}              \cr
\scriptstyle{16} &   f_g                               \cr
}
\end{equation}

\begin{equation}\notag
\eta_i \equiv
\bordermatrix {
                 &                                              \cr
\scriptstyle{0}  &   c_{2,i}                                    \cr
\scriptstyle{1}  &   0                                          \cr
\scriptstyle{2}  &   p_{cfix, i}^{(l)}                          \cr
\scriptstyle{3}  &   0                                          \cr
\scriptstyle{4}  &   (-V_{i,j})_{j \in \mathcal{P}}^\text{re}   \cr
\scriptstyle{5}  &   \chi^{(l)}_{i,ref}                         \cr
\scriptstyle{6}  &   0                                          \cr
\scriptstyle{7}  &   (-V_{i,j})_{j \in \mathcal{A}}^\text{re}   \cr
\scriptstyle{8}  &   \chi^{(l)}_{i}                             \cr
\scriptstyle{9}  &   ns                                         \cr
\scriptstyle{10} &  0                                           \cr
\scriptstyle{11} &  0                                           \cr
\scriptstyle{12} &  0                                           \cr
}
\end{equation}

\begin{table}[h!b!p!]
\caption{Set naming convention}
\centering
\begin{tabular}{ c l }
\hline
Set              & Description                                                \\
\hline
$\mathcal{B}$    & rigid bodies                                               \\
$\mathcal{M}$    & muscle fibers                                              \\
$\mathcal{A}$    & anchor points                                              \\
$\mathcal{J}$    & anchored joints                                            \\
$\mathcal{P}$    & contact points                                             \\
$\mathcal{D}$    & degenerated muscle fibers ($\mathcal{D} \in \mathcal{M}$)  \\
\hline
\end{tabular}
\end{table}

\end{landscape}



\begin{landscape}
\subsection{Agonist muscles}
\begin{scriptsize}
\begin{tabular}{|l|l|p{4cm}|p{4cm}|l|}
  % after \\: \hline or \cline{col1-col2} \cline{col3-col4} ...
  muscle group      & muscle                    & origin                                                  & insertion                                                 & antagonist         \\
  \hline
  \multirow{3}{*}{gluteal muscles}   & gluteus maximus           & gluteal surface of ilium, lumbar fascia                 & gluteal tuberosity of the femur                           &                    \\
                    & gluteus medius            & gluteal surface of ilium                                & greater trochanter of the femur                           &                    \\
                    & gluteal minimus           & gluteal surface of ilium                                & greater trochanter of the femur                           &                    \\
  \hline
  \multirow{3}{*}{hamstring}         & biceps femoris            & ischial tuberosity(long head), femur(short head)        & head of the fibula                                        & rectus femoris     \\
                    & semitendinosus            & ischial tuberosity                                      & medial surface of tibia                                   & rectus femoris     \\
                    & semimembranosus           & ischial tuberosity                                      & lateral side of the head of tibia                         & rectus femoris     \\
  \hline
  \multirow{4}{*}{quadriceps femoris}& vastus medialis           & femur                                                   & patellar, tibial tuberosity                               & hamstring          \\
                    & vastus intermedius        & antero/lateral femur                                    & patellar, tibial tuberosity                               & hamstring          \\
                    & vastus lateralis          & greater trochanter of the femur                         & patellar, tibial tuberosity                               & hamstring          \\
                    & rectus femoris            & anterior inferior iliac spine                           & patellar                                                  & hamstring          \\
  \hline
                    & adductor magnus           & ischial tuberosity                                      & femur                                                     &                    \\
                    & gastronecmius             & lateral/medial condyle of femur                         & tendo calcaneous (achilles tendon)                        & anterior tibialis  \\
                    & soleus                    & fibula                                                  & tendo calcaneous                                          & anterior tibialis  \\
                    & anterior tibialis         & body of tibia                                           & medial(first) cuneiform                                   & posterior tibialis, gastrocnemius, soleus \\
                    & posterior tibialis        & tibia, fibula                                           & medial(first) cuneiform                                   & anterior tibialis  \\
                    & extensor digitorum longus & lateral condyle of tibia                                & middle and distal phalanges(toe) of lateral four digits   & flexor digitorum longus \\
                    & flexor digitorum longus   & posterior of tibia                                      & base of the distal phalanges(toe) of the four lesser toes & extensor digitorum longus \\
\end{tabular}
\end{scriptsize}
\end{landscape}


%%%%%%%%%%%%%%%%%%%%%%%%%%%%%%%%%%%%%%%%%%%%%%%%
%%%%%%%%%%%%%%%%%%%%%%%%%%%%%%%%%%%%%%%%%%%%%%%%
%%%%%%%%%%%%%%%%%%%%%%%%%%%%%%%%%%%%%%%%%%%%%%%%

\begin{landscape}
\begin{equation}\notag
\left[\begin{array}{ccccccccccccccccccccc}
0 & \mathbf{M}/h^2                 & -\mathbf{1}                & \cdots   &  -\mathbf{1}                              &                           &            &                                          &               &        &                &                     &        &                                 &    &         &    & \mathbf{0} & 0  & 0 & 0  \\
0 &                                & -\mathbf{1}                & \cdots   &  -\mathbf{1}                              & \mathbf{Q}_1              & \cdots     &  \mathbf{Q}_{|\mathcal{P}|}              &               &        &                &                     &        &                                 &    &         &    & \mathbf{0} & 0  & 0 & 0  \\
0 &    \mathbf{1}                  &                            &          &                                           &\mathbf{0}                 &            &                                          &               &        &                &                     &        &                                 &    &         &    & \mathbf{1} & 0  & 0 & 0  \\
0 & \mathbf{c}_1                   &                            &          &                                           &\mathbf{0}                 &            &                                          &               &        &                &                     &        &                                 &    &         &    & \mathbf{0} & -1 & 0 & 0  \\
0 & \mathbf{0}                     &                            &          &                                           & \mathbf{c}_2              &            &                                          &               &        &                &                     &        &                                 &    &         &    & \mathbf{0} & 0  & 0 & -1 \\
0 &   \mathbf{Z}_1                 &                            &          &                                           &                           &            &                                          &  -\mathbf{1}  &        &                &                     &        &                                 &    &         &    &            & 0  & 0 &    \\
0 &      \vdots                    &                            &          &                                           &                           &            &                                          &               &\ddots  &                &                     &        &                                 &    &         &    &            & 0  & 0 &   \\
0 &   \mathbf{Z}_{|\mathcal{P}|}   &                            &          &                                           &                           &            &                                          &               &        & -\mathbf{1}    &                     &        &                                 &    &         &    &            & 0  & 0 &   \\
0 &                                & -\mathbf{c}_{4,1}          &          &                                           &                           &            &                                          & \mathbf{c}_3  &        &                & -1                  &        &                                 &    &         &    &            & 0  & 0 &   \\
0 &                                &                            & \ddots   &                                           &                           &            &                                          &               & \ddots &                &                     & \ddots &                                 &    &         &    &            & 0  & 0 &   \\
0 &                                &                            &          & -\mathbf{c}_{4,|\mathcal{P}|}             &                           &            &                                          &               &        & \mathbf{c}_3   &                     &        & -1                              &    &         &    &            & 0  & 0 &   \\
0 &                                &  \sqrt{\mathbf{c}_{4,1}}   &          &                                           &                           &            &                                          &               &        &                & 1/2\sqrt{c_{fcz,1}} &        &                                 & -1 &         &    &            & 0  & 0 &   \\
0 &                                &                            & \ddots   &                                           &                           &            &                                          &               &        &                &                     & \ddots &                                 &    &  \ddots &    &            & 0  & 0 &   \\
0 &                                &                            &          & \sqrt{\mathbf{c}_{4,|\mathcal{P}|}}       &                           &            &                                          &               &        &                &                     &        & 1/2\sqrt{c_{fcz,|\mathcal{P}|}} &    &         & -1 &            & 0  & 0 &
\end{array}\right]
\left[\begin{array}{c}
\epsilon \\ \chi^{(l+1)} \\
f_{c,1} \\ \vdots \\ f_{c,|\mathcal{P}|} \\
c_{c,1} \\ \vdots \\ c_{c,|\mathcal{P}|} \\
\tilde{p}^{(l+1)}_{c,1} \\ \vdots \\ \tilde{p}^{(l+1)}_{c,|\mathcal{P}|} \\
\tilde{q}_{1} \\ \vdots \\ \tilde{q}_{|\mathcal{P}|} \\
\tilde{c}_{lcp,1} \\ \vdots \\ \tilde{c}_{lcp,|\mathcal{P}|} \\
\Delta\chi \\
\epsilon_{fric} \\ \mu f_{c,z}
\end{array}\right]
\end{equation}

\end{landscape}

\end{document}
